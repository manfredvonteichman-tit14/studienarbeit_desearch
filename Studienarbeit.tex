%%%%%%%%%%%%%%%%%%%%%%%%%%%%%%%%%%%%%%%%%%%%%%%%%%%%%%%%%%%%%%%%%%%%%%%%%%%%%
%%                                                                         %%
%%                   LaTeX Vorlage für die Studenten der                   %%
%%              Dualen Hochschule Baden-Württemberg Ravensburg             %%
%%                                                                         %%
%%  Die Vorlage orientiert sich an den Gestaltungsrichtlinien der DHBW RV  %%
%%                                                                         %%
%%                                                                         %%
%% Ersteller:        Markus Schutz (WI06)                                  %%
%% letzten Änderung: 26. November 2013                                     %%
%%                                                                         %%
%%                                                                         %%
%% Wichtiger Hinweis zur Verwendung dieser Vorlage:                        %%
%%                                                                         %%
%% Damit die Vorlage verwendet und die PDF richtig und vollständig erzeugt %%
%% werden kann bedarf des manuellen Aufrufs von makeindex. Diese können    %%
%% optional auch direkt im Editor eingerichtet werden.                     %%
%% TeXnicCenter (Ausgabe -> Ausgabeprofile definieren... (Alt + F7)        %%
%%               LaTeX => PDF -> Nachbearbeitung)                          %%
%%                                                                         %%
%% (1) Stichwortverzeichnis (falls verwendet)                              %%
%% makeindex -g -s styles\stichwortverzeichnis.ist vorlage                 %%
%% (2) Abkürzungsverzeichnis                                               %%
%% makeindex vorlage.nlo -s cl.ist -o vorlage.nls                     %%
%% (3) Glossar (falls verwendet)                                           %%
%% makeindex -s vorlage.ist -t vorlage.glg -o vorlage.gls vorlage.glo      %%
%%                                                                         %%
%% Bei der Erzeugung der PDF Datei (Anwendung des LaTeX => PDF Ausgabe-    %%
%% profiles) werden das Abkürzungsverzeichnis, das Stichwortverzeichnis    %%
%% und das Glossar jetzt richtig erzeugt. In Verbindung mit TeXnicCenter   %%
%% kann es beim automatisierten Aufruf von makeindex zu Probleme kommen.   %%
%% Ein manueller Aufruf funktioniert dagegen immer.                        %%
%%                                                                         %%
%% Wichtiger Hinweis:                                                      %%
%%                                                                         %%
%% Keine Änderungen an den Dateien im Verzeichnis "pages" vornehmen. Für   %%
%% die Arbeit beziehen sich alle Änderungen auf diese Datei und die        %%
%% Dateien im Verzeichnis "chapter".                                       %%
%% Für die Erstellung des Literaturverzeichnises empfiehlt sich die Ver-   %%
%% wendung von JabRef (http://jabref.sourceforge.net). Die Datei ist unter %%
%% dem Namen literatur.bib im Verzeichnis "literatur" zu speichern.        %%
%%                                                                         %%
%% Zur sinnvollen Nutzung dieser Vorlage empfiehlt es sich, die Dokus zu   %%
%% den eingebundenen Paketen durchzulesen. Sie sind im doc-Verzeichnis der %%
%% MiKTeX-Installation zu finden.                                          %%
%%                                                                         %%
%% Enthaltene Titelblätter:                                                %%
%%   - Seminararbeit                                                       %%
%%   - Projektarbeit                                                       %%
%%   - Bachelorarbeit                                                      %%
%%                                                                         %%
%%%%%%%%%%%%%%%%%%%%%%%%%%%%%%%%%%%%%%%%%%%%%%%%%%%%%%%%%%%%%%%%%%%%%%%%%%%%%

\documentclass[a4paper,12pt]{article}                                         % Schriftgröße, Layout, Papierformat, Art des Dokumentes
\usepackage[left=3cm,right=2cm,top=2cm,bottom=2cm,includehead]{geometry}      % Einstellungen der Seitenränder
\usepackage{ngerman}                                                          % neue Rechtschreibung
\usepackage[main=ngerman,english]{babel} 
\usepackage{lmodern}                                                  % deutsche Silbentrennung
\usepackage[utf8]{inputenc} 
%\usepackage[ansinew]{inputenc}                                                  % Umlaute
\usepackage[T1]{fontenc}
\usepackage{textcomp}
\usepackage[hyperfootnotes=false]{hyperref}                                   % pfd-Output [Fußnoten nicht verlinken]
\usepackage[nottoc]{tocbibind}                                                % Inhaltsverzeichniserweiterung (Inhaltsverzeichnis selbst ausblenden)
\usepackage{makeidx}                                                          % Index
\usepackage[intoc]{nomencl}                                                   % Abkürzungsverzeichnis
\usepackage{fancyhdr}                                                         % Fancy Header
\usepackage[round]{natbib}                                                    % Zitate (Erweiterung für Literaturverzeichnis)
\usepackage{amsmath}                                                          % Zurücksetzen der Tabellen- und Abbildungsnummerierung je Sektion
\usepackage[labelfont=bf,aboveskip=1mm]{caption}                              % Bild- und Tabellenunterschrift (fett)
\usepackage{setspace}                                                         % Zeilenabstand (vor footmisc laden!)
\usepackage[bottom,multiple,hang,marginal]{footmisc}                          % Fußnoten [Ausrichtung unten, Trennung durch Seperator bei mehreren Fußnoten]
\usepackage{graphicx}                                                         % Grafiken
\usepackage{tabularx}                                                         % erweiterte Tabellen
\usepackage{longtable}                                                        % mehrseitige Tabellen
\usepackage{color}                                                            % Farben
\usepackage{enumitem}                                                         % Befehl setlist (Zeilenabstand für itemize Umgebung auf 1 setzen)
\usepackage[formats]{listings}                                                         % Quelltexte
\usepackage{zref}                                                             % Verweise (Anhangsverweise)
\usepackage[toc,style=altlist,translate=false]{glossaries}                    % Glossar (nach hyperref, 
%\usepackage{glossaries-babel}                                                 % Glossar: Übersetzung im TOC
\usepackage{url}
\usepackage[autostyle]{csquotes}
\usepackage{framed}
\usepackage{lipsum}
\usepackage[final]{pdfpages}
\usepackage{float}															% Platzierung der Bilder
\definecolor{shadecolor}{gray}{0.9}
\definecolor{gray}{rgb}{0.4,0.4,0.4}
\definecolor{darkblue}{rgb}{0.0,0.0,0.6}
\definecolor{cyan}{rgb}{0.0,0.6,0.6}




%%%%%%%%%%%%%%%%%%%%%%%%%%%%%%%%%%%%%%%%%%%%%%%%%%%%%%%%%%%%%%%%%%%%%%%%%%%%%
%%                                                                         %%
%% \/   \/      Bitte hier die Änderungen zur Arbeit vornehmen     \/   \/ %%
%%                                                                         %%
%%%%%%%%%%%%%%%%%%%%%%%%%%%%%%%%%%%%%%%%%%%%%%%%%%%%%%%%%%%%%%%%%%%%%%%%%%%%%

%%%%%%%%%%%%%%%%%%%%%%% Definitionen bzgl. der Arbeit %%%%%%%%%%%%%%%%%%%%%%%
\def\myType{3}           % [0=Seminararbeit|1=Projektarbeit|2=Bachelorarbeit|3=Studienarbeit]

\def\myTopic{Prototypische Implementierung eines Dozentenfinders mittels Bluetooth-Technologie}
\def\mySubTopic{ zur Machbarkeits-Evaluierung des “DeSearch”-Projekts }
\def\myAutor{Philipp Stehle und Lina Hirschoff}
\def\myMatNr{5316233 und 4125931}
\def\myCompany{SAP SE}
\def\myCompanyAddressStreet{Dornierstraße 3}
\def\myCompanyAddressCity{88677 Markdorf}
\def\myProf{Andreas Judt}
\def\myEndDate{15.07.2016}
\def\myEditDate{KW 40/2015 - KW 28/2016}

%%%%%%%%%%%%%%%%%%%% Folgende Angaben für: Seminararbeit %%%%%%%%%%%%%%%%%%%%
\def\myVorlesung{Name der Vorlesung}

%%%%%%%%%%%%%%%%%%%% Folgende Angaben für: Projektarbeit %%%%%%%%%%%%%%%%%%%%
\def\myProjNumber{4}         % [1|2]
\def\myPraxPhase{3}                       % [1|2|3]

%%%%%%%%%%%%%%%%%%%%%%%%%%%%%%%%%%%%%%%%%%%%%%%%%%%%%%%%%%%%%%%%%%%%%%%%%%%%%
%%                                                                         %%
%% /\   /\         Ab hier keine Änderungen mehr vornehmen         /\   /\ %%
%%                                                                         %%
%%%%%%%%%%%%%%%%%%%%%%%%%%%%%%%%%%%%%%%%%%%%%%%%%%%%%%%%%%%%%%%%%%%%%%%%%%%%%

%%%%%%%%%%%%%%%%%%%%%%%% Eigene Farbwerte definieren %%%%%%%%%%%%%%%%%%%%%%%%
\definecolor{boxgray}{gray}{0.9}         % Hintergrundfarbe für Zitatboxen
\definecolor{commentgray}{gray}{0.5}     % Grau für Kommentare in Quelltexten
\definecolor{darkgreen}{rgb}{0,.5,0}     % Grün für Strings in Quelltexten
\definecolor{purple}{rgb}{0.44, 0.16, 0.39} %Lila für JS-Keywords in Quelltexten


%%%%%%%%%%%%%%%%%%%%%%%% Eigene Kommandos definieren %%%%%%%%%%%%%%%%%%%%%%%%
% Definition von \gqq{#1: text}: Text in Anführungszeichen
\newcommand{\gqq}[1]{\glqq #1\grqq}

% Definition von \footref{#1: label}
% Verweis auf bereits existierende Fußnoten mittels
\providecommand*{\footref}[1]{
	\begingroup
		\unrestored@protected@xdef\@thefnmark{\ref{#1}}
	\endgroup
\@footnotemark}

% Definition von \mypageref{#1: label}
% Kombination aus \ref{#1: label} und \pageref{#1: label}
\newcommand{\mypageref}[1]{\ref{#1} auf Seite \pageref{#1}}

% Definition von \myboxquote{#1: text}
% grau hinterlegte Quotation-Umgebung (für Zitate)
\newcommand{\myboxquote}[1]{
	\begin{quotation}
		\colorbox{boxgray}{\parbox{0.78\textwidth}{#1}}
	\end{quotation}
	\vspace*{1mm}
}

\makeatletter
\zref@newprop*{appsec}{}
\zref@addprop{main}{appsec}

% Definition von \applabel{#1: label}{#2: text}
% von \appsec{#1: text}{#2: label} zur Erzeugung des Labels verwendet)
\def\applabel#1#2{%
	\zref@setcurrent{appsec}{#2}%   
	\zref@wrapper@immediate{\zref@label{#1}}%
}

% Definition von \appref{#1: label}
% anstelle \ref{#1: label} zum referenzieren von Anhängen verwenden)
\def\appref#1{%
	\hyperref[#1]{\zref@extract{#1}{appsec}}%
}
\makeatother

% Definition von \appsection{#1: text}{#2: label}
% Ersetzt \section{#1: text} und \label{#2: label} für Anhänge)
\newcommand{\theappsection}[1]{Anhang \Alph{section}:~\protect #1}
\newcommand{\appsection}[2]{
	\addtocounter{section}{1}
	\phantomsection
	\addcontentsline{toc}{section}{\theappsection{#1}}
	\markboth{\theappsection{#1}}{}

	\section*{\theappsection{#1}}
	\applabel{#2}{Anhang \Alph{section}}
	\label{#2}
}

\newcommand\tab[1][1cm]{\hspace*{#1}}

%%%%%%%%%%%%% Index, Abkürzungsverzeichnis und Glossar erstellen %%%%%%%%%%%%
\makeindex
\makenomenclature
\makeglossaries

% Festlegung der Art der Zitierung (Havardmethode: Abkuerzung Autor + Jahr) %
\bibliographystyle{dinat}

%%%%%%%%%%%%%%%%%%%%%%%%%%%%%%% PDF-Optionen %%%%%%%%%%%%%%%%%%%%%%%%%%%%%%%%
\hypersetup{
	bookmarksopen=false,
	bookmarksnumbered=true,
	bookmarksopenlevel=0,
	pdftitle=\myTopic,
	pdfsubject=\myTopic,
	pdfauthor=\myAutor,
	pdfborder=0 0 0,
	pdfstartview=Fit,
	pdfpagelayout=SinglePage
}

%%%%%%%%%%%%%%%%%%%%%%%%%%%% Kopf- und Fußzeile %%%%%%%%%%%%%%%%%%%%%%%%%%%%%
\pagestyle{fancy}
\fancyhf{}
\fancyhead[R]{\thepage}                         % Kopfzeile rechts bzw. außen
\renewcommand{\headrulewidth}{0.5pt}            % Kopfzeile rechts bzw. außen

%%%%%%%%%%%%%%%%%%%%%%%%% Layout und Beschriftungen %%%%%%%%%%%%%%%%%%%%%%%%%
\onehalfspacing                % Zeilenabstand: 1.5 (Standard: 1.2)
\setlist{noitemsep}            % Zeilenabstand für items auf 1 setzen

\addto\captionsngerman{        % Tabllen- und Abbildungsunterschriften ändern
  \renewcommand{\figurename}{Abb.}
  \renewcommand{\tablename}{Tab.}
}
\numberwithin{table}{section}                               % Tabellennummerierung je Sektion zurücksetzen
\numberwithin{figure}{section}                              % Abbildungsnummerierung je Sektion zurücksetzen
\renewcommand{\thetable}{\arabic{section}.\arabic{table}}   % Tabellennummerierung mit Section
\renewcommand{\thefigure}{\arabic{section}.\arabic{figure}} % Abbildungsnummerierung mit Section
\renewcommand{\thefootnote}{\arabic{footnote}}              % Sektionsbezeichnung von Fußnoten entfernen

\renewcommand{\multfootsep}{, }                             % Mehrere Fußnoten durch ", " trennen

%%%%%%%%%%%%%%%%%%%%%%%%%%%%%%% Listingstyle %%%%%%%%%%%%%%%%%%%%%%%%%%%%%%%%
\lstset{
%	basicstyle=\ttfamily\small
%	basicstyle=\scriptsize\ttfamily\bfseries,
%	basicstyle=\fontfamily{pcr}\selectfont\scriptsize,
	%basicstyle=\itshape\small,
 basicstyle=\footnotesize\ttfamily,
	commentstyle=\color{commentgray}\textit,
	showstringspaces=false,
	stringstyle=\color{darkgreen},
	keywordstyle=\color{blue},
%	numbers=left,
%	numberstyle=\tiny,
%	stepnumber=1,
%	numbersep=10pt,
	tabsize=2,
	showspaces=false,
	keepspaces=true
	showtabs=false,
	fontadjust=true,
	frame=single,
%	backgroundcolor=\color{boxgray},
	captionpos=b,
	linewidth=1.0\linewidth,
	xleftmargin=0\linewidth,
	breaklines=true,
	aboveskip=16pt
}
\lstdefinelanguage{XML}
{
  morestring=[b]",
  morestring=[s]{>}{<},
  morecomment=[s]{<?}{?>},
  stringstyle=\color{black},
  identifierstyle=\color{darkblue},
  keywordstyle=\color{cyan},
  morekeywords={xmlns,version,type}% list your attributes here
}
\lstdefinelanguage{JavaScript}{
  keywords={break, case, catch, continue, debugger, default, delete, do, else, false, finally, for, function, if, in, instanceof, new, null, return, switch, this, throw, true, try, typeof, var, void, while, with},
  morecomment=[l]{//},
  morecomment=[s]{/*}{*/},
  morestring=[b]',
  morestring=[b]",
  ndkeywords={class, export, boolean, throw, implements, import, this},
  keywordstyle=\color{blue}\bfseries,
  ndkeywordstyle=\color{darkgray}\bfseries,
  identifierstyle=\color{black},
  commentstyle=\color{purple}\ttfamily,
  stringstyle=\color{darkgreen}\ttfamily,
  sensitive=true
}

        
%%%%%%%%%%%%%%%%%%%%%%%%%%%%%%%%%%%%%%%%%%%%%%%%%%%%%%%%%%%%%%%%%%%%%%%%%%%%%
%%                                                                         %%
%% \/   \/      Bitte hier die Änderungen zur Arbeit vornehmen     \/   \/ %%
%%                                                                         %%
%%%%%%%%%%%%%%%%%%%%%%%%%%%%%%%%%%%%%%%%%%%%%%%%%%%%%%%%%%%%%%%%%%%%%%%%%%%%%

%Seiten und Kapitel einbinden
\begin{document}
	\pagenumbering{Roman}
	% Das Titelblatt wird automatisch ausgewählt. Keine Änderung hier
	\ifcase\myType
		\begin{titlepage}
	\begin{center}
		\vspace*{2cm}
		\LARGE\bf\myTopic\\
		\Large\rm\mySubTopic\\
		\vspace*{3cm}
		\bf Seminararbeit zur Vorlesung\\
		\myVorlesung\\
		\normalsize\rm
		\vspace*{1cm}
		für die\\
		Prüfung zum Bachelor of Engineering\\
		\vspace*{1cm}
		an der Fakultät für Technik\\
		im Studiengang Informationstechnik\\
		\vspace*{1cm}
		an der\\
		DHBW Ravensburg\\
		Campus Friedrichshafen
		\vfill
	\end{center}
	\begin{tabular}{ll}
		Verfasser:&\myAutor\\
		Ausbildungsbetrieb:&\myCompany\\
		Anschrift:&\myCompanyAddressStreet\\
		&\myCompanyAddressCity\\
	%	Wiss. Betreuer:&\myProf\\
		Abgabedatum:&\myEndDate\\
	\end{tabular}
\end{titlepage}
\newpage
\setcounter{page}{2}

	\or
		\begin{titlepage}

\begin{table}
\begin{tabular}{p{0.4\textwidth}p{0.4\textwidth}}
\includegraphics[width=4cm]{images/sap_logo.png} &
\hfill \includegraphics[width=4cm]{images/dhbw_logo.jpg} \\
\end{tabular}
\end{table}

	\begin{center}
		\vspace*{1cm}
		\LARGE\bf\myTopic\\
		%\Large\rm\mySubTopic\\
		\vspace*{1cm}
		\bf Projektarbeit\\
		
		\normalsize\rm
		 im \myPraxPhase. Studienjahr \\
		\vspace*{1cm}
		
		Duale Hochschule Baden-Württemberg\\
		\vspace*{0.5cm}
		TIT13 Informationstechnik\\
		\vspace*{1cm}
		von\\
		 \myAutor\\
		\vspace*{1cm}
		
		\vfill
	\end{center}
	\begin{tabular}{ll}
		Abgabe:&\myEndDate\\
		Bearbeitungszeitraum:&\myEditDate\\
		Matrikelnummber:&\myMatNr\\
		Partnerunternehmen:&\myCompany\\
		Addresse:&\myCompanyAddressStreet\\
					&\myCompanyAddressCity\\
		\\
		Betreuer:&\myProf\\
		
	\end{tabular}
	\newline
	\vspace*{1cm}
%	\newline
%	\begin{tabularx}{\textwidth}{l@{\extracolsep\fill}r}
%	  Unterschrift des verantwortlichen Ausbilders&\\
%	  (oder des Personalverantwortlichen)&\rule{6cm}{0.3mm}\\
%	\end{tabularx}
\end{titlepage}
\newpage
\setcounter{page}{2}

	\or
		\include{pages/10_titel_bachelor}
	\or
		\begin{titlepage}

\begin{table}
\begin{tabular}{p{0.4\textwidth}p{0.4\textwidth}}
%\includegraphics[width=4cm]{images/sap_logo.png} &
\hfill \includegraphics[width=6cm]{images/dhbw_logo.jpg} \\
\end{tabular}
\end{table}

	\begin{center}
		\vspace*{1cm}
		\LARGE\bf\myTopic\\
		\Large\rm\mySubTopic\\
		\vspace*{1cm}
		\bf Studienarbeit\\
		
		\normalsize\rm
		 im 3. Studienjahr \\
		\vspace*{1cm}
		
		Duale Hochschule Baden-Württemberg\\
		\vspace*{0.5cm}
		TIT13 Informationstechnik\\
		\vspace*{1cm}
		von\\
		 \myAutor\\
		\vspace*{1cm}
		
		\vfill
	\end{center}
	\begin{tabular}{ll}
		Abgabe:&\myEndDate\\
		Bearbeitungszeitraum:&\myEditDate\\
		Matrikelnummer:&\myMatNr\\
		
		Betreuer:&\myProf\\
		
	\end{tabular}
	\newline
	\vspace*{1cm}
%	\newline
%	\begin{tabularx}{\textwidth}{l@{\extracolsep\fill}r}
%	  Unterschrift des verantwortlichen Ausbilders&\\
%	  (oder des Personalverantwortlichen)&\rule{6cm}{0.3mm}\\
%	\end{tabularx}
\end{titlepage}
\newpage
\setcounter{page}{2}

	\else
	\fi
	
	\pagestyle{fancy}
	\thispagestyle{empty}
\addcontentsline{toc}{section}{Selbständigkeitserklärung}
\begin{center}
	\vspace*{2cm}
	\Huge\bf Selbständigkeitserklärung\\
	\vspace*{2cm}
	\normalsize\rm
	gemäß Ziffer 1.2.3 der Anlage 1 zu §§ 3, 4, und 5  der „Studien- und Prüfungsordnung DHBW Technik“ vom 29. September 2015.\\
	\vspace*{1cm}
	\normalsize\rm
	Ich versichere hiermit, dass ich meine \ifcase\myType Seminararbeit \or Projektarbeit \or Bachelorarbeit \or Studienarbeit \else\fi mit dem Thema: \\
	\vspace*{1cm}
	\Large\bf\myTopic\\
	\Large\rm\mySubTopic\\
	\vspace*{1cm}
	\normalsize\rm
	selbständig verfasst und keine anderen als die angegebenen\\Quellen und Hilfsmittel benutzt habe. Ich versichere zudem, dass die eingereichte elektronische Fassung mit der gedruckten Fassung übereinstimmt.\\
	\vspace*{4cm}
	\begin{tabularx}{\textwidth}{l@{\extracolsep\fill}r}
  	\rule{7cm}{0.3mm}\\
	\end{tabularx}
	\begin{tabularx}{\textwidth}{*{2}{>{\arraybackslash}X}}
	  Ort, Datum\\
	\end{tabularx}
	
	\vfill
	\begin{tabularx}{\textwidth}{l@{\extracolsep\fill}r}
		\rule{7cm}{0.3mm}&\rule{7.55cm}{0.3mm}\\
	\end{tabularx}
	\begin{tabularx}{\textwidth}{*{2}{>{\arraybackslash}X}}
		Unterschrift&Unterschrift\\
	\end{tabularx}
\end{center}

%	\begin{titlepage}
	\begin{center}
		\vspace*{1cm}
		\Huge\bf Sperrvermerk\\
		\vspace*{2cm}
		\normalsize\rm
		\begin{quotation}
			\parbox{0.8\textwidth}{Der Inhalt dieser \ifcase\myType Seminararbeit \or Projektarbeit \or Bachelorarbeit\else\fi darf weder als Ganzes noch in Auszügen Personen außerhalb des Prüfungsprozesses und des Evaluationsverfahrens zugänglich gemacht werden, sofern keine anders lautende Genehmigung der Ausbildungsstätte vorliegt. Ausnahmen bedürfen der schriftlichen Genehmigung der Firma \myCompany, \myCompanyAddressStreet, \myCompanyAddressCity.}
		\end{quotation}
		\vspace*{1cm}
%		\begin{quotation}
%		  \parbox{0.8\textwidth}{
%		  \begin{tabularx}{0.78\textwidth}{l@{\extracolsep\fill}l}
%				\rule{4cm}{0.3mm}&\rule{4cm}{0.3mm}\\
%	    	Ort, Datum&Unterschrift
%			\end{tabularx}}
%		\end{quotation}
	\end{center}
\end{titlepage}
\newpage
\setcounter{page}{3}

%	\include{00_signaturen}
	\section*{Abstract}
Die zunehmend alternde Gesellschaft stellt uns vor eine steigende Anzahl an Demenzpatienten. Das \enquote{Altern zu Hause} als Wunsch vieler Patienten bedeutet eine große Herausforderung für Angehörige und Pflegende. Ein Problem der Demenz ist die sogenannte Weglauftendenz, die bei Angehörigen große Panik oder Angst vor dem Vermisstenfall auslösen kann. Diese Problemstellung war ein Anreiz dafür, das iCare-Projekt mit dem DeSearch-System ins Leben zu rufen.\\
Das iCare-Projekt, das sich mit den Möglichkeiten von IoT-Techniken im Kontext des \enquote{Alterns zu Hause} beschäftigt, erforscht Technologien, die Alternden und Pflegenden den Alltag erleichtern. Unter dieser Zielsetzung ist auch das DeSearch-System als Teil von iCare entstanden, das die Positionsbestimmung von an Demenz erkrankten Menschen in Angriff nimmt. Im Vermisstenfall kann eine Suche im DeSearch-System ausgelöst und die Position der vermissten Person über Bluetooth bestimmt werden. Die Zielsetzung dieser Projektarbeit war die Umsetzung eines Prototypen von DeSearch, welcher die Suche nach Dozenten und Studenten auf dem Campus der DHBW in Friedrichshafen ermöglicht. 
Für die Umsetzung des Projekts wurden Hardware-Entscheidungen getroffen, Prototypen implementiert und das System in der DHBW Friedrichshafen ausgiebig getestet. In Zukunft soll der Prototyp in Zusammenarbeit mit der Internationalen Bodenseehochschule weiter verbessert sowie in einem Pflegezentrum installiert und getestet werden.
\begin{center}
	\rule{0.3\textwidth}{0.4mm}
\end{center}
Rising numbers of patients with dementia demand new strategies and solutions in elder care. Many patients wish to \enquote{age in place}, at home. One problem of people with dementia is their tendency to run away and get lost. To solve this problem, the \enquote{iCare}-Project was initiated. This Project drives research on how IoT-Technologies help in the daily routine of caregivers or relatives. The \enquote{DeSearch}-System as a part of iCare targets the location determination of persons with dementia. In a missing person case, a search can be triggered in the DeSearch-System, which locates the missing person via Bluetooth. The objective of this student research project was the implementation and realization of a prototype which can be tested in the area of DHBW Friedrichshafen. The execution included hardware decisions, software implementation and testing in the campus area. In the foreseeable future, the prototype will be transferred to a retirement home, where productive testing will be continued. 
	\include{pages/12_inhaltsverzeichnis}
	\include{pages/13_abkuerzungsverzeichnis}

	\include{pages/14_glossar}
	\include{pages/15_abbildungsverzeichnis}
	\include{pages/16_tabellenverzeichnis}
	\include{pages/17_listingsverzeichnis}

	% Kapitel
	\pagestyle{fancy}
	\fancyhead[L]{\nouppercase{\leftmark}}                               % Kopfzeile links bzw. innen
	\pagenumbering{arabic}
	
	
	
\section{Einleitung}
\subsection{Hinführung}
In nahezu jedem Land der Erde wächst der Anteil der Menschen über 60 schneller an als jede andere Altersgruppe der Bevölkerung. Mit der alternden Bevölkerung steigt auch die Zahl der Demenz-Erkrankungen. In Deutschland lebten 2014 1,5 Millionen Demenzkranke, jährlich treten ca. 300.000 Neuerkrankungen auf. Aufgrund des demografischen Gesellschaftswandels kommt es zu mehr Neuerkrankungen als Sterbefällen, wodurch die Zahl der Demenzpatienten kontinuierlich ansteigt \citep[Vgl.][]{demenz}.\\
Zeitgleich zeichnet sich eine weitere Entwicklung in unserer Gesellschaft ab: Das \enquote{Internet of Things} wird allgegenwärtig. Wir umgeben uns mit elektronischen Geräten, Sensoren, Überwachungssystemen, und lassen uns gerne und oft von smarten Gegenständen und dem Internet helfen. Möglichkeiten, die vor kurzer Zeit noch nur als Gedankenexperiment existierten, lassen sich nun umsetzen. Das Zusammenspiel von Sensoren und Datenanalysen und die Kommunikation vieler Geräte untereinander rückt auch bei medizinischen Anwendungen immer mehr in den Fokus. Überwachung, Diagnose und medizinische Versorgung können durch das IoT unterstützt und vereinfacht werden. In der folgenden Arbeit sollen IoT-Technologien und die Problemstellungen bei der Pflege von Demenzpatienten zusammengeführt werden.
\subsection{Problemstellung}
Bei an Demenz erkrankten Menschen besteht das Risiko einer Weglauftendenz. Das bedeutet, dass diese ohne das Wissen von Angehörigen oder Pflegern beschließen, ihr Haus oder das Pflegeheim zu verlassen. Vom Zeitpunkt des Verschwindens bis jemand dieses bemerkt können kritische Stunden vergehen, in denen die Betroffenen in einen Bus einsteigen, stürzen oder unterkühlen können. Auch das Verschwinden im Pflegeheim selbst kann schon zur Gefahr werden, wenn sich die Betroffenen beispielsweise stundenlang im Keller aufhalten. Ist ein Demenzpatient verschwunden, bricht bei Angehörigen oder Pflegern Panik aus. Suchaktionen, nicht selten mit der Polizei zusammen, werden gestartet. Bei der Überlegung, wie man solchen Notfallsituationen vorbeugen kann, stehen die Pflegenden vor einem Dilemma. Jeder Patient hat das Recht auf uneingeschränkte Bewegunsfreiheit, durch Freiheitsentzug könnte das Problem des Weglaufens jedoch gelöst werden. Auch Sicherungssysteme wie Ortungschips sind rechtlich umstritten. Laut Pflege-Qualitätssicherungsgesetz gelten sie als freiheitsentziehende Maßnahme und bedürfen der richterlichen Genehmigung \citep[Vgl.][]{pqsg}. Andererseits existieren auch Urteile, die das Anbringen von Ortungssystemen ohne gerichtliche Zustimmung erlauben\footnote{OLG Brandenburg, 19.01.2006 - 11 Wx 59/05\\AG Coesfeld, 31.08.2007 - Az. 9 XVII 214/06}. Die Schwierigkeit besteht also darin, die dementen Personen vor den Gefahren des Weglaufens zu schützen, ohne ihnen dabei die Freiheit zu entziehen. Diese Problemstellung war ein Anreiz dafür, das iCare-Projekt mit dem DeSearch-System ins Leben zu rufen.
\subsection{Zielsetzung}
 Ziel von DeSearch ist es, den Standort der betroffenen Menschen im Vermisstenfall diskret bestimmen zu können. Dabei sollen der Aufenthaltsort der Personen keinesfalls dauerhaft getrackt werden, da dies rechtlich nicht tragbar wäre. Stattdessen soll bei einem gemeldeten Vermissten eine Suche ausgelöst werden, worauf das DeSearch-System den Standort der Person über Bluetooth bestimmt. Dies soll durch den Einsatz von kleinen Bluetooth-Marken, die unbemerkt in die Kleidung eingenäht werden können, realisiert werden. Die Marken werden von sogenannten DeSearch-Boxen erkannt, wenn sie sich in Sendereichweite der Boxen befinden, und können so einem Standort zugeordnet werden. Eine Zentrale, die vom Pflegepersonal bedient werden kann, dient dem Starten von Personensuchen und dem Anzeigen der Ergebnisse über eine Web-Oberfläche. Wird die Person nicht als vermisst gemeldet, können keinerlei Rückschlüsse auf deren Aufenthaltsort gezogen werden.
\\Zielsetzung dieser Projektarbeit ist es, einen lauffähigen Prototypen der DeSearch-Box zu bauen und mit den Bluetooth-Marken zu verknüpfen. Das erste Testsystem soll im Umfeld der DHBW Friedrichshafen aufgesetzt werden, und als “Dozentenfinder” getestet werden. Ziel dabei ist es, einen Dozenten mit Bluetooth-Marken und das DHBW-Gebäude mit den DeSearch-Boxen auszustatten. Es soll möglich sein, den Dozenten anhand der aufgestellten DeSearch-Boxen lokalisieren zu können und eine Aufzeichnung seiner Aufenthaltsorte zu generieren.
Im weiteren Verlauf soll der Projektumfang so ausgeweitet werden, dass das System in einem Seniorenzentrum der Evangelischen Altenheimat Bondorf im größeren Umfang getestet werden kann.

\subsection{Methodisches Vorgehen zur Problemlösung}
Die einzelnen Schritte der Studienarbeit werden sein:
\begin{itemize}
\item Recherche zur Tauglichkeit verschiedener Hardware-Varianten
\item Hardware-Entscheidung
\item Hardware-Beschaffung (kleine Stückzahl zum Anfertigen eines Prototypen)
\item Bau eines Prototypen
\item Aufbau der Systemarchitektur und Vernetzung(im Hüttenworkshop in Zusammenarbeit mit dem ganzen Kurs)
\item Aufbau des Systems in der DHBW Friedrichshafen
\item Testläufe mit Dozentenfinder
\item Aufbau und Einrichten des Testsystems im Seniorenzentrum Altenheimat Bondorf
\end{itemize}



\subsection{Erwartungen an die Lösungsergebnisse}
Folgende Anforderungen sollten durch den ersten Prototypen erfüllt sein:
\begin{itemize}
	\item  Absicherung der Daten und der DeSearch-Boxen, beispielsweise durch Zertifikate, zum Schutz gegen Manipulation
	\item Authentifizierung der Administratoren auf den DeSearch-Boxen und der Zentrale 
	\item Periodisches Anfragen der DeSearch-Boxen bei der Zentrale, welche Marken momentan gesucht werden
	\item Meldung der DeSearch-Box bei der Zentrale, wenn eine Marken gesehen wurde mit: Marken-ID, Box-ID, Signalstärke, Zeitstempel
	\item Starten einer Suche über die Benutzeroberfläche der Zentrale
	\item Anzeigen der Suchergebnisse auf der Benutzeroberfläche
	\item Eine Person liegt in der Datenbank vor mit: Vorname, Nachname, ID, Anmerkungen zur Person (beispielweise Medikamente, Verhalten, Berduerfnisse,...), gesucht-Flag
	\item Zuordnung einer Personen-ID zu den MAC-Adressen der Marken, sodass einer Person mehrere Marken zugeordnet werden können
	\item Hinzufügen, Bearbeiten und Entfernen von Mitarbeitern, Personen und Marken in der Datenbank, Bedienung über die Benutzeroberfläche der Zentrale
\end{itemize}

\nomenclature{ID}{\textbf{Id}entifikator}
\nomenclature{MAC-Adresse}{\textbf{M}edia-\textbf{A}ccess-\textbf{C}ontrol-Adresse}

\subsection{Überprüfung der Ergebnisse}
Die Überprüfung der Ergebnisse soll zum einen anhand der gestellten Anforderungen erfolgen. Die ausführlichen Geschäfts- und Systemfälle werden einzeln durchgespielt und auf Erfüllung überprüft. Eine Auflistung der vor Projektbeginn spezifizierten Systemfälle ist in Anhang \ref{anh:fälle} zu sehen. Zusätzlich zu diesen Spezifikationen wird die Installation in der Testumgebung und der ausführliche Testbetrieb zur Überprüfung der Ergebnisse genutzt. Details zur Evaluierung der Ergebnisse finden sich in Kapitel \ref{sec:evaluierung}.










	\section{Problemstellung und Zielsetzung}\label{sec:definitionen}
\subsection{Das Problem der Weglauftendenz}
Eine sogenannte Weglauftendenz, also die Neigung zum \enquote{Ausbüxen} von zu Hause oder aus der Pflegeeinrichtung, entwickeln sehr viele Demente Menschen. Oftmals gibt es einen Grund wie Langeweile oder das Gefühl, am falschen Ort zu sein. Patienten in einem Pflegeheim haben häufig die Idee, ihr Elternhaus oder die Arbeitsstelle aufzusuchen. Es wird unterschieden zwischen ziellosen Umherwandern (Rastlosigkeit) und zielgerichtetem Aufbrechen, weswegen oftmals das Wort \enquote{Hinlauftendenz} präferiert wird \citep[Vgl.][]{hinlauf}. Das Problem dabei ist, dass der Kranke in der Regel vergessen hat, warum oder wohin er eigentlich losgegangen ist \citep[Vgl.][]{dgk}. Dabeit gehen die Patienten das Risiko der Dehydration, Unterzuckerung, oder unterbliebener Medikation ein. Im Winter besteht zudem Erfrierungsgefahr. In der Pflege von Menschen mit Weglauftendenz herrscht eine Spannung zwischen zwei sich widersprechenden Grundsätzen: Einerseits hat jeder Mensch das Recht, sich frei zu bewegen. Gleichzeitig sollen die dementen Menschen vor Gesundheitsgefahren beschützt werden \citep[Vgl.][]{pqsg}.
\subsection{Related Work: IoT-Anwendungen in Medizin und Pflege}\label{ssec:rel.work}
Das Welt des Internet of Things (IoT)) ist bereits auf die Pflege als lukratives Anwendungsfeld gestoßen.
\nomenclature{IoT}{\textbf{I}nternet \textbf{o}f \textbf{T}hings}
 Dementsprechend sind bereits einige Lösungen auf dem Markt, die durch die Nutzung von Technologien wie Bluetooth, GPS, Druck- und Beschleunigungssensoren versuchen, die Arbeit von Pflegenden zu erleichtern. Millionen von Sensoren, Aktoren, Mikrocontrollern und Kommunikationsgeräten erfassen rund um die Uhr Patientendaten, werten Muster und Auffälligkeiten aus und können so individuelle Versorgung, Diagnose oder Überwachung von pflegebedürftigen Menschen bereitstellen \citep[Vgl.][]{digikey}.
Das deutsche Bundesministerium für Bildung und Forschung (BMBF) hat eigens für die Weiterentwicklung solcher Themen eine Initiative mit dem Namen \enquote{Pflegeinnovationen 2020} ins Leben gerufen. Unter Anderem werden durch diese Initiative Forschungsprojekte gefördert, die Innovationen der Mensch-Technik-Interaktion entwickeln und dadurch Pflegende unterstützen \citep[Vgl.][]{bmbf}. Fertige Produkte und Systeme sind bereits einige auf dem Markt.
Beispielsweise hat ein 13-Jähriger ein Überwachungssystem für seinen Großvater entwickelt, das in seine Socken integriert ist. Damit kann über Drucksensoren registriert werden, wann er nachts aufsteht \citep[Vgl.][]{spiegel-alzheimer}. Nach dem ähnlichen Prinzip funktionieren auch druckempflindliche Fußmatten, die schon weit verbreitet eingesetzt werden, beispielsweise als Bettvorleger oder am Zimmerausgang.  Ein Experiment der Intel-GE Care Innovations beispielsweise setzt den Fokus auf das \enquote{Altern vor Ort} und erforscht hierbei Technologien und Produkte, die es Senioren ermöglichen, so lange wie möglich selbständig zu Hause zu leben. Der erste Prototyp dieser Firma war der sogenannte \enquote{Magic Carpet} - ein mit Sensoren ausgestatteter Fußboden, der in der Wohnung einer älteren Person installiert werden kann. Zunächst werden typische Bewegungsmuster und Alltagsrituale über den Fußboden erfasst, wie zum Beispiel der Gang in die Küche jeden Morgen um 7 Uhr. Anschließend wird das Verhalten der Person regelmäßig auf Abnormalitäten überprüft. Bei einer kritischen Abweichung werden Angehörige oder Ärzte informiert \citep[Vgl.][S.49]{bigdata}.\\
Auch Wearables, also am Körper tragbare elektronische Helfer, sind bereits im Pflege-Bereich angekommen. Die Firma Eurotronik bietet beispielsweise ein System an, bei dem Demente mit Weglauftendenz über ein Funk-Armband geortet werden können. Zudem wird ein Alarm ausgelöst, wenn sie ihre erlaubten Aufenthaltsbereiche verlassen \citep[Vgl.][]{eurotronik}. Etwas diskreter ist der Einsatz von GPS-Armbanduhren. Diese werden von den Patienten nicht unbedingt als Ortungsarmband erkannt, was eine gesteigerte Benutzerakzeptanz nach sich zieht. Ein Beispiel hierfür sind die Uhren der Firma DS Vega \citep[Vgl.][]{ds-vega}.
Einen ähnlichen Ansatz der Ortung hat die Firma Aetrex gewählt, die einen GPS-Tracker unauffällig in spezielle Schuhe einbaut. Die Schuhe senden alle 30 Minuten die Position des Trägers an einen Account, den die Angehörigen oder die Pfleger einsehen können. Die Akkulaufzeit der GPS-Sender beträgt allerdings maximal 48 Stunden \citep[Vgl.][]{aetrex}.
http://winfwiki.wi-fom.de/index.php/Einsatz\_von\_Wearables\_für\_Demenzerkrankte\\




\subsection{Projektübersicht iCare}
Das Projekt iCare ist ein vor einigen Jahren als Wettbewerbsidee ins Leben gerufenes Projekt mit dem Ziel, Menschen das \enquote{Altern zu Hause} zu ermöglichen und dabei Angehörige und Pfleger zu unterstützen.  Das iCare-Projekt ist über die studentische Projektarbeit hinaus bereits ein von der internationalen Bodenseehochschule gefördertes Forschungsprojekt \citep[Vgl.][]{icare-dhbw}.
Das DeSearch-System als ein Teil von iCare beschäftigt sich mit der selben Problemstellung wie die bereits in Kapitel \ref{ssec:rel.work} erwähnten Systeme: demente Personen mit Weglauftendenz. Allerdings soll das Ortungssystem von DeSearch dabei nicht vom Patienten erkannt werden. Ein Armband oder eine GPS-Uhr trägt immer das Risiko, von der dementen Person als elektronische Fußfessel angesehen zu werden. Sobald die Person die Uhr oder das Armband abnimmt, ist das System außer Kraft gesetzt. Einige Hersteller lösen dieses Problem mit Verschlüssen, die abgeschlossen werden können oder nur mit 2 Händen geöffnet werden können. Doch auch solche Systeme kann ein Patient zerstören, wenn er nicht mit der Ortung einverstanden ist. Beispielsweise steigen die Patienten dann mit dem Armband in die Badewanne oder greifen zu Scheren oder Messern, um die Uhr loszuwerden. Das Problem der Nutzerakzeptanz soll beim DeSearch-System dadurch gelöst werden, dass die Bluetooth-Marken, die zur Ortung dienen, unsichtbar in die Kleidung der Patienten eingenäht werden können. Auch wasserdicht und waschmaschinenfest sollen die Marken sein, sodass sie einfach mit der Kleidung mitgewaschen werden können. Ein weiteres Problem von GPS-Tracking-Systemen ist die Akkulaufzeit der Geräte. Die GPS-Schuhe müssen alle 48 Stunden geladen werden, die GPS-Uhr hat eine maximale Akkulaufzeit von 72 Stunden, abhängig von der Mobilfunknetz-Abdeckung. Mit der Auswahl von Bluetooth als Technologie für die Ortung haben die DeSearch-Marken den Vorteil, dass sie deutlich länger mit einer kleinen Knopfzelle betrieben werden können. Der Bluetooth-Low-Energy Standard ist hierfür maßgeblich (siehe Kapitel \ref{sssec:BLE}).

\subsection{Zielsetzung}
Das übergeordnete Ziel von iCare ist es, alternden Menschen so lange wie möglich das Leben zu Hause in einer vertrauten Umgebung zu bieten. Durch innovative Technologien, die Vernetzung von Geräten und Computerprogramme soll dieses Ziel verwirklicht werden und den Pflegern die Arbeit erleichtert werden. Ein Teil davon ist das DeSearch-System, das demente Patienten mit einer Weglauftendenz schützen und den Angehörigen oder Pflegern bei der Betreuung helfen soll.
Zielsetzung des DeSearch-Projektes ist die Entwicklung eines Ortungssystems, das von den Patienten unbemerkt in die Kleidung eingenäht wird. Die Anschaffungskosten sollen im Vergleich zu bisher auf dem Markt existenten Systemen niedrig sein, damit möglichst viele Kleidungsstücke mit Marken ausgestattet werden soll. Die Ortung von vermissten Personen soll über die Bluetooth-Low-Energy Technologie realisiert werden, damit die Marken eine möglichst lange Akkulaufzeit erreichen. Zudem darf die Position der Personen nicht dauerhaft aufgezeichnet werden, sondern nur im Vermisstenfall einmalig ermittelt werden. Eine Speicherung der Aufenthaltsorte wird nicht vorgenommen. Die Ortung erfolgt über eine Erkennung der Bluetooth-Marken von sogenannten DeSearch-Boxen, die in der Umgebung installiert werden. Diese melden den Fund dann an eine Zentrale, die die Ergebnisse auf einer Web-Oberfläche darstellt. Im Rahmen dieser Projektarbeit soll ein lauffähiger Prototyp der Zentrale, der DeSearch-Boxen und der Web-Oberfläche erstellt werden, der dann in einer Testumgebung an der dualen Hochschule getestet werden soll.

	\section{Methodisches Vorgehen zur Problemlösung}

%%%%%%%%%%%%%%%%%%%%%%%%%%%%%%%%%%%%%%%%%%%%%%%%%%%%%%%%%%%%%%%%%%%%%%%%%%%%%%%%%%%%%%%%%%%%%%
\subsection{Aktueller Stand der Technologie}

\subsubsection{Bluetooth Low Energy}\label{sssec:BLE}

\subsubsection{Raspberry Pi}
Für die Zentrale und die DeSearch-Boxen ist zunächst eine Hardware-Entscheidung notwendig. Benötigt werden Mikrocontroller oder Mikrocomputer mit folgenden Eigenschaften:
\begin{itemize}
	\item W-LAN Verbindung von der Box zur Zentrale möglich
	\item Bluetooth-fähige Box zur Markenerkennung
	\item Zentrale muss als Server fungieren und HTTP-Requests senden und verarbeiten können
	\item Datenbank-Installation zur Datenhaltung notwendig
\end{itemize}
Zudem sollen die Kosten pro Gerät so gering wie möglich gehalten werden. \\
Der Rasberry Pi ist ein Mikrocomputer mit einer Grundfläche, die etwas größer ist als eine Kreditkarte. Die Anschaffungskosten liegen ohne Zubehör bei etwa 42 €. Für diese geringen Anschaffungskosten erhält man einen vollwertigen, Linux-Basierten Computer mit einer ARM-CPU, W-LAN und Bluetooth-Schnittstelle. Im Gegensatz zu Mikrocontrollern ist der Raspberry Pi leistungsfähiger und läuft stabiler \citep[Vgl.][S.35ff.]{raspi}. Auf einem Arduino beispielweise kann nur C++-Code in einer Endlos-Schleife ausgeführt werden. Der Pi hingegen bietet die Möglichkeit, Bash-Skripte, Python-Code, Datenbank-Anfragen und Webserver gleichzeitig auszuführen. Die Entscheidung für Raspberry Pi ist auch aufgrund der relativ niedrigen Anschaffungskosten für einen kompletten Rechner mit Betriebssystem gefallen. Für die Entwickler ist das Betriebssystem Linux zudem am einfachsten zu bedienen. Für das Projekt werden Pakete mit SD-Karte, W-LAN und Bluetooth-Dongles, Netzteil und Gehäuse für ca. 75 € pro Paket angeschafft.
\subsubsection{Authentifizierungstechnologie}
\subsubsection{Software-Rollout mit apt-get}
\subsubsection{PostgreSQL}


\subsection{Umsetzung der Anforderungen}

\subsubsection{Vernetzung der DeSearch-Boxen und Infrastruktur}
\subsubsection{User-Authentifizierung}
\subsubsection{Erkennen der Marken und Kommunikation zur Zentrale}
\subsubsection{Datenhaltung mittels PostgreSQL}
\subsubsection{Administrative Benutzeroberfläche}
\subsubsection{Installation der DeSearch-Boxen in der Testumgebung}
Die Raspberry Pi's werden in der DHBW Ravensburg am Campus Friedrichshafen zum Testbetrieb installiert. Dabei fungiert ein Pi als Zentrale und die anderen als DeSearch-Boxen, die die Funde an die Zentrale melden. In Tabelle \ref{tab:pis} ist eine Übersicht über alle installierten Raspberry Pi's aufgelistet.
Dazu wurde uns vom Netzwerkadministrator der DHBW die Erlaubnis eingeräumt, die vorhandene Infrastruktur zu verwenden. Daran geknüpft waren allerdings folgende Bedingungen bzw. Einschränkungen:
\begin{itemize}
	\item Zugriffe aus dem Internet sind nicht erlaubt.
	\item Alle Pi's außer der Zentrale (Slaves) verbinden sich über WLAN unter Verwendung unserer Zugangsdaten mit dem Netzwerk.
	\item Die Zentrale bekommt einen LAN Anschluss und eine feste IP Adresse plus DNS Eintrag zugewiesen.
	\item Ein Verbindungsaufbau zu den Slaves ist nicht möglich - jegliche Kommunikation muss von ihnen initiiert werden.
\end{itemize}
Die letzte Bedingung stelle eine Herausforderung dar, da damit auch kein SSH-Zugriff auf die DeSearch-Boxen möglich ist, der benötigt wird um die DeSearch Anwendung zu aktualisieren, starten oder überwachen.
Dieses Problem konnte gelöst werden, da mit der Zentrale ein Gerät im Netzwerk ist, das für alle anderen Pi's und auch die zur Entwicklung und Wartung verwendeten Laptops erreichbar ist, kann darüber ein Tunnel aufgebaut werden.
Zum Aufbauen des Tunnels hat sich die Verwendung der SSH Portforwarding Funktion angeboten, da alle beteiligten Geräte bereits für die Verwendung von SSH eingerichtet waren. Eine schematische Darstellung der Vorgehensweise finden sie in Abbildung \ref{fig:tunnel}.
Zum Verbindungsaufbau wurde ein technischer Benutzer mit dem Namen "tunnel" auf der Zentrale angelegt, der sich über SSH und Zertifikat Authentifizierung einloggen kann.
Jeder Slave hat einen RSA Schlüssel erhalten, den er zum Login verwenden kann, und über einen Eintrag ein der Crontabelle wird sichergestellt, dass der Tunnel immer offen gehalten wird.
Der Ausgangsport des Tunnels ist dabei für jede DeSearch-Box eindeutig. Soll nun eine Verbindung zu einer Box hergestellt werden, so muss erst eine SSH Verbindung zur Zentrale aufgebaut werden und dort eine SSH Verbindung zum Eingang des Tunnels (localhost plus Tunnel Ausgangport) hergestellt werden.
\begin{table}[h]
	\begin{tabular}{ | p{2,5cm} | p{2,5cm} | p{4cm} | p{6cm} |}
		\hline
		\textbf{Pi-Nummer} & \textbf{Funktion} & \textbf{Installationsort} &  \textbf{Erreichbarkeit} \\ \hline
		3 & Zentrale & Büro Herr Judt & statische IP 141.68.30.39 oder \mbox{Judt-Master.it.ba-ravensburg.de} \\ \hline
		5 & DeSearch-Box & Haupteingang oberhalb der Treppe & Tunnel über Zentrale, Port 19005 \\ \hline
		
	\end{tabular}
	\caption{Übersicht der Raspberry Pi's mit Funktion, Installationsort und Erreichbarkeit}
	\label{tab:pis}
\end{table}

\begin{figure}
	\centering
	\includegraphics[width=\textwidth]{./images/tunnel.png}
	\caption[Schematische Darstellung der Verwendung des SSH Tunnels]{\textbf{Schematische Darstellung der Verwendung des SSH Tunnels} - Beispielhaft ist hier eine Verbindung vom Administrator zur DeSearch-Box 6 dargestellt (roter Pfeil).}
	\label{fig:tunnel}
\end{figure} 


	\section{Evaluierung der Lösungsergebnisse}\label{sec:evaluierung}
Die Lösungsergebnisse sollen im Folgenden Kapitel anhand fester Kriterien überprüft werden. Ein Kriterienkatalog wurde aus den Geschäfts- und Systemfällen entwickelt, die als Ergebnis aus einer studentischen Veranstaltung zu Projektbeginn entstanden sind (siehe Anhang \ref{anh:fälle}).
Anhand der Kriterien in Tabelle \ref{tab:kriterien} soll die Problemlösung überprüft werden. Alle Kriterien werden in der Testumgebung an der DHBW Friedrichshafen durchgespielt und anschließend bewertet.
\begin{center}
	\begin{longtable}{ | p{2,5cm} | p{5cm} | p{5cm} | p{2,5cm} |}
	 \hline
	 \textbf{Kriterium} & \textbf{Beschreibung} & \textbf{Erwartetes Ergebnis} & \textbf{Erfüllungs-grad} \\ \hline
	 Vermisster Patient wird erfasst & Ein Patient wurde in der Zentrale als vermisst markiert und die entsprechende Marke befindet sich in Reichweite einer DeSearch-Box & Die Box erkennt die entsprechende Marke und meldet sich bei der Zentrale & erfüllt; Im Test je nach Bluetooth-Gerät (Marke, Smartphone) unterschiedlich zuverlässig erkannt \\ \hline
	 Andere Marken werden ignoriert & Ein Patient wurde in der Zentrale als vermisst markiert und Marken von anderen Patienten befinden sich in Reichweite einer DeSearch-Box & Die Box erkennt, dass die Marken in Reichweite nicht vermisst werden und meldet nichts zurück & erfüllt; Im Test kein einziger false positive \\ 
	 \hline
	 DeSearch-Box wird konfiguriert & Eine neue DeSearch-Box soll konfiguriert bzw. eine bestehende Box soll angepasst werden & Konfigurierte und im System verbundene DeSearch-Box & erfüllt; Siehe Kapitel \ref{sssec:schritte} \\ 
	 \hline
	 DeSearch-Box wird befestigt & Eine DeSearch-Box soll am Einsatzort befestigt werden & Vor Vandalismus und Diebstahl gesicherte, im Netzwerk erreichbare Box mit permanenter Stromversorgung & nicht zutreffend für Prototyp, Anforderung an Produktivsystem \\ 
	 \hline
	 temporärer Ausfall DeSearch-Box & Ein unerwarteter Ausfall der Box tritt zeitweise auf & Nach Reboot der Box soll sich diese selbständig wieder im System anmelden und den Scan-Vorgang wieder aufnehmen & erfüllt durch Services und cronjob, siehe Kapitel \ref{sssec:services}\\ 
	 \hline
	 permanenter Ausfall DeSearch-Box & Ein unerwarteter Ausfall der Box tritt permanent auf, die Box sendet keine alive-Meldungen mehr & Das System muss den Ausfall feststellen, sodass die Box manuell wieder im System reaktiviert werden kann & erfüllt; In der UI kann Ausfall festgestellt werden(siehe Kapitel \ref{sssec:ui}) \\ 
	 \hline
	 DeSearch-Box fordert gesucht-Liste an & Die Box fordert periodisch die aktuelle Liste der gesuchten Marken bei der Zentrale an & Die Zentrale übermittelt der Box die aktuelle Liste der gesuchten Marken & erfüllt, Anfrage alle 10 Sekunden \\ 
	 \hline
	 App fordert gesucht-Liste an & Die Benutzeroberfläche fordert beim Start der Applikation die aktuelle Liste der gesuchten Personen an & Die Zentrale übermittelt der App die aktuelle Liste der gesuchten Personen & erfüllt, siehe Kapitel \ref{sssec:ui}\\ 
	 \hline
	 Mitarbeiter deaktiviert Marke & Eine gesuchte Marke wurde gefunden, aber die gesuchte Person trägt dieses Kleidungsstück momentan nicht & In der App wird die Marke deaktiviert, die Zentrale entfernt die Marke von der gesucht-Liste & erfüllt, siehe Kapitel \ref{sssec:ui} \\ 
	 \hline
	 Person wurde gefunden & Ein Mitarbeiter beendet die Personensuche in der App & Die Zentrale entfernt alle entsprechenden Marken von der gesucht-Liste & erfüllt, siehe Kapitel \ref{sssec:ui}\\ 
	 \hline
	 
	 \caption{Erfüllungskriterien zur Überprüfung der Problemlösung}
	
	\label{tab:kriterien}
	\end{longtable}
	
\end{center}


	\section{Schlussteil}
\subsection{Zusammenfassung}
\subsection{Ausblick}

	
	
	% Abschluss
	\bibliography{literatur/PA4_lina_literatur}
\newpage

	%\include{pages/19_index}
		
	% Anhang
	\renewcommand{\thetable}{\Alph{section}.\arabic{table}}              % Tabellennummerierung mit Section
	\renewcommand{\thefigure}{\Alph{section}.\arabic{figure}}            % Abbildungsnummerierung mit Section
	\renewcommand{\thelstlisting}{\Alph{section}.\arabic{lstlisting}}    % Listingsnummerierung mit Section
	
	\begin{appendix}
	
\includepdf[pages=1,scale=0.8,clip,pagecommand={\section{Geschäfts- und Systemfälle zur Überprüfung der Ergebnisse}\label{anh:fälle}}]{src/Geschaefts-Systemfaelle.pdf}

\includepdf[pages=2-,scale=0.9,clip,pagecommand={}]{src/Geschaefts-Systemfaelle.pdf}
\newpage
%\section{Master-Detail-Layout}\label{anh:master-detail}

	\end{appendix}
	
	
\end{document}

%%%%%%%%%%%%%%%%%%%%%%%%%%%%%%%%%%%%%%%%%%%%%%%%%%%%%%%%%%%%%%%%%%%%%%%%%%%%%
%%                                                                         %%
%% /\   /\         Ab hier keine Änderungen mehr vornehmen         /\   /\ %%
%%                                                                         %%
%%%%%%%%%%%%%%%%%%%%%%%%%%%%%%%%%%%%%%%%%%%%%%%%%%%%%%%%%%%%%%%%%%%%%%%%%%%%%