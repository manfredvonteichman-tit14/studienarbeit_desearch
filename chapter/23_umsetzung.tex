\section{Methodisches Vorgehen zur Problemlösung}

%%%%%%%%%%%%%%%%%%%%%%%%%%%%%%%%%%%%%%%%%%%%%%%%%%%%%%%%%%%%%%%%%%%%%%%%%%%%%%%%%%%%%%%%%%%%%%
\subsection{Aktueller Stand der Technologie}

\subsubsection{Bluetooth Low Energy}\label{sssec:BLE}
\subsubsection{Authentifizierungstechnologie}
\subsubsection{Software-Rollout mit apt-get}
\subsubsection{PostgreSQL}
\subsubsection{Raspberry Pi}

\subsection{Umsetzung der Anforderungen}

\subsubsection{Vernetzung der DeSearch-Boxen und Infrastruktur}
\subsubsection{User-Authentifizierung}
\subsubsection{Erkennen der Marken und Kommunikation zur Zentrale}
\subsubsection{Datenhaltung mittels PostgreSQL}
\subsubsection{Administrative Benutzeroberfläche}
\subsubsection{Installation der DeSearch-Boxen in der Testumgebung}
Die Raspberry Pi's werden in der DHBW Ravensburg am Campus Friedrichshafen zum Testbetrieb installiert. Dabei fungiert ein Pi als Zentrale und die anderen als DeSearch-Boxen, die die Funde an die Zentrale melden. In Tabelle \ref{tab:pis} ist eine Übersicht über alle installierten Raspberry Pi's aufgelistet. Ein Zugriff über das Netzwerk ist nur von den Boxen in Richtung Zentrale erlaubt, deswegen muss für den SSH-Zugriff auf die DeSearch-Boxen ein Tunnel über die Zentrale zu den Boxen aufgebaut werden. Hierfür wurde jeder Box ein Port zugewiesen, über den die Zentrale dann als Reverse Proxy die Verbindung herstellt.
\begin{table}[h]
	\begin{tabular}{ | p{2,5cm} | p{2,5cm} | p{4cm} | p{6cm} |}
		\hline
		\textbf{Pi-Nummer} & \textbf{Funktion} & \textbf{Installationsort} &  \textbf{Erreichbarkeit} \\ \hline
		3 & Zentrale & Büro Herr Judt & statische IP 141.68.30.39 oder \mbox{Judt-Master.it.ba-ravensburg.de} \\ \hline
		5 & DeSearch-Box & Haupteingang oberhalb der Treppe & Tunnel über Zentrale, Port 19005 \\ \hline
		
	\end{tabular}
	\caption{Übersicht der Raspberry Pi's mit Funktion, Installationsort und Erreichbarkeit}
	\label{tab:pis}
	
\end{table}
