\section{Methodisches Vorgehen zur Problemlösung}

%%%%%%%%%%%%%%%%%%%%%%%%%%%%%%%%%%%%%%%%%%%%%%%%%%%%%%%%%%%%%%%%%%%%%%%%%%%%%%%%%%%%%%%%%%%%%%
\subsection{Aktueller Stand der Technologie}

\subsubsection{Bluetooth-Technologie}\label{sssec:BLE}
Bluetooth ist eine drahtlose Datenübertragunsgtechnologie, die basierend auf der Funktechnik Verbindungen zwischen Geräten über eine kurze Distanz ermöglicht. Im engen Nahbereich gilt Bluetooth als der Kommunikationsstandard \citep[Vgl.][S. 133]{mobil-sicher}. Mit dem Bluetooth Low Energy (BLE \nomenclature{BLE}{\textbf{B}luetooth \textbf{L}ow \textbf{E}nergy}) Standard, der 2009 verabscheidet wurde der Stromverbrauch für eine Bluetooth-Verbindung in den Endgeräten drastisch reduziert. Dies gelang unter anderem durch kürzere Verbindungsaufbauzeiten und Schlafphasen (Standby) zwischen den Sendezyklen. Eine vergleichende Untersuchung zum Energieverbrauch von Bluetooth und BLE veröffentlichten \cite{ble-energy}. 


\subsubsection{Raspberry Pi}
Für die Zentrale und die DeSearch-Boxen ist zunächst eine Hardware-Entscheidung notwendig. Benötigt werden Mikrocontroller oder Mikrocomputer mit folgenden Eigenschaften:
\begin{itemize}
	\item W-LAN Verbindung von der Box zur Zentrale möglich
	\item Bluetooth-fähige Box zur Markenerkennung
	\item Zentrale muss als Server fungieren und HTTP-Requests senden und verarbeiten können
	\item Datenbank-Installation zur Datenhaltung notwendig
\end{itemize}
Zudem sollen die Kosten pro Gerät so gering wie möglich gehalten werden. \\
Der Rasberry Pi ist ein Mikrocomputer mit einer Grundfläche, die etwas größer ist als eine Kreditkarte. Die Anschaffungskosten liegen ohne Zubehör bei etwa 42 €. Für diese geringen Anschaffungskosten erhält man einen vollwertigen, Linux-Basierten Computer mit einer ARM-CPU, W-LAN und Bluetooth-Schnittstelle. Im Gegensatz zu Mikrocontrollern ist der Raspberry Pi leistungsfähiger und läuft stabiler \citep[Vgl.][S.35ff.]{raspi}. Auf einem Arduino beispielweise kann nur C++-Code in einer Endlos-Schleife ausgeführt werden. Der Pi hingegen bietet die Möglichkeit, Bash-Skripte, Python-Code, Datenbank-Anfragen und Webserver gleichzeitig auszuführen. Die Entscheidung für Raspberry Pi ist auch aufgrund der relativ niedrigen Anschaffungskosten für einen kompletten Rechner mit Betriebssystem gefallen. Für die Entwickler ist das Betriebssystem Linux zudem am einfachsten zu bedienen. Für das Projekt werden Pakete mit SD-Karte, W-LAN und Bluetooth-Dongles, Netzteil und Gehäuse für ca. 75 € pro Paket angeschafft.
\subsubsection{Authentifizierungstechnologie}
\subsubsection{Software-Rollout mit apt und dpgk}
Zum Installieren von Software wird unter Debain basierten System (dazu gehört neben Ubuntu und natürlich Debian selbst auch das Standardbetriebssystem des Raspberry Pi: Rasbian) das Programm "dpkg" verwendet.
Dazu muss eine ".deb" Datei vorliegen. Statt dem manuellen Download, der auch bei jedem Update erneut durchgeführt werden müsste, lässt sich zum Verteilen der deb Dateien apt verwenden.
Apt selbst nimmt keine Installation vor, es kümmert sich lediglich um die Beschaffung der aktuellsten deb Dateien und deren Abhängigkeiten (ebenfalls im deb Format) und reicht diese zur Installation an dpkg weiter.
Neue Programme lassen sich mit "apt-get install <packetname>" installieren.
Dazu hält sich apt eine Liste mit allen verfügbaren Programmen in einem lokalen Cache vor. Um diese Liste zu aktualisieren kann der Befehl "apt-get update" verwendet werden.
Dabei werden alle in der Datei "/etc/sources.list" bzw. dem Ordner "/etc/sources.list.d/" spezifizierten Quellen abgefragt.
Die Quellen werden dabei als URL spezifiziert und können unterschiedlichste Protokolle erfordern.
Beispielsweise HTTP, HTTPS, FTP oder lokale Datenträger.
Zum aktualisieren aller installierten Programme (nach dem ausführen von "apt-get update") wird der Befehl "apt-get upgrade" verwendet. Dabei wird die Softwareliste im Cache mit den installierten Programmen abgeglichen, ist eine aktuellere Version verfügbar wird deren deb Paket heruntergeladen und mittels dpkg installiert.
\newline\textbf{apt}\newline
Zum Aktualisieren der Softwareliste lädt apt zunächst die Release Datei von der Quelle runter, diese ist mit gnupg signiert und enthält Informationen über die Packetquelle wie die unterstützen CPU Architekturen.
Außerdem sind Hash-Summen der Packages Dateien enthalten, da diese nicht gnupg signiert sind.
Danach wir die Package Datei heruntergeladen. Diese Datei ist spezifisch für eine Architektur und kann in einem Repository mehrfach vorkommen.
Sie enthält die Namen aller verfügbaren Pakete dieses Repositories sowie deren Abhängigkeiten, Beschreibung, Pfad und Hash-Werte der zugehörigen deb Datei sowie weitere Metadaten.
\newline\textbf{dpkg}\newline
dpkg installiert, konfiguriert und deinstalliert deb Pakete.
Eine deb Datei enthält in einem Archiv alle Dateien die das zu installierende Programm zur Ausführung benötigt.
Außerdem sind in einem weiteren Archiv verschiedene Kontrolldateien enthalten, eine davon enthält Metadaten wie Namen des Pakets, Abhängigkeiten und der Beschreibung.
Daneben sind (optional) Skripte enthalten die ausgeführt werden bevor oder nachdem das Paket Installiert, Deinstalliert oder Aktualisiert wurde. 

\subsubsection{PostgreSQL}
PostgreSQL ist ein objektrelationales Datenbank-Management-System (DBMS), das aus dem Projekt POSTGRES der kalifornischen Universität Berkeley heraus entstanden ist. Der objektrelationale Ansatz bedeutet eine Erweiterung der relationalen DBMS um die Konzepte der Objektorientierung. Somit können in objektrelationalen DBMS beispielsweise komplexe Datenstrukturen mit Attributen definiert werden oder Vererbungen vorgenommen werden \citep[Vgl.][S. 135f.]{datenbanken}. In anderen DBMS muss hierfür \enquote{object relational mapping} (ORM) hinzugezogen werden, um Objektstrukturen auf relationale Tabellen abzubilden \citep[Vgl.][S. 426]{balzert} Viele Konzepte, die in POSTGRES neu waren, wurden später in kommerzielle DBMS übernommen. PostgreSQL ist die Open-Source-Variante von POSTGRES und unterstützt einen großen Teil des SQL-Standards. Zudem bietet es Features wie komplexe Queries, Fremdschlüsselbeziehungen, Trigger, updatefähige Views und transaktionale Integrität. Außerdem kann der Benutzer PostgreSQL beliebig durch neue Datentypen, Funktionen, Operatoren oder Aggregationsfunktionen erweitern. Aufgrund der Open-Source-Lizenz kann PostgreSQL von jedem verwendet, verbessert oder verteilt werden
\nomenclature{DBMS}{\textbf{D}aten\textbf{b}ank \textbf{M}anagement \textbf{S}ystem}
\citep[Vgl.][preface, S. lxvi]{postgres}. Für das DeSearch-Projekt eignet sich PostgreSQL vor allem wegen seiner Open-Source-Lizenz, was die Kosten für das Projekt gering hält. Zudem ist die Möglichkeit, eigene Datentypen definieren zu können, von Vorteil. Da die Datenbank auf dem Zentral-Raspberry verwendet wird, der meist über SSH in der Konsole bedient wird, ist PostgreSQL einfach zu bedienen, da es über einen Konsolenclienten verfügt.

\subsubsection{Services unter Linux mit systemd}\label{sssec:systemd}
\textbf{systemd} ist ein Init-Tool unter Linux, das als neuere Alternative zu sysvinit und upstart gilt. Es ist für das Starten von Services und Mounten von Hard Disks beim boot zuständig. Systemd verwaltet Hintergrunddienste(Daemons) und bietet ein Event-gesteuertes Starten von diesen, analog zum Runlevel-Prinzip von sysvinit. Dies bedeutet, dass bestimmte Gruppen von Services an bestimmten Punkten beim Boot gestartet werden, zum Beispiel sobald die GUI verfügbar ist oder sobald eine aktive Netzwerkverbindung besteht. In Tabelle \ref{tab:lvl} ist ein Vergleich zwischen den sysvinit-runlevels und den systemd-targets zu sehen. 
\begin{table}[h]
	\begin{tabular}{|p{2,5cm}|p{3,5cm} |p{8,5cm} | }
		\hline
		\textbf{sysvinit Runlevel} & \textbf{systemd-target} & \textbf{Bedeutung}\\ \hline
		0 & poweroff.target & System-Shutdown	\\ \hline
		1, single & rescue.target & Single-User-Modus \\ \hline
		2,4 & multi-user.target & Benutzerdefinierte runlevels, per default identisch mit runlevel 3 \\ \hline
		3 & multi-user.target & Multi-User-Umgebung, ohne graphische Benutzer\-oberfläche (Login über Konsole möglich)\\ \hline
		5 & graphical.target & Multi-User-Umgebung, graphische Benutzer\-oberfläche \\ \hline
		6 & reboot.target & System-Reboot \\ \hline
		emergency & emergency.target & Emergency Shell\\ \hline
		
	\end{tabular}
	\caption{Übersicht über die targets von systemd, in Anlehnung an \cite{fedora}}
	\label{tab:lvl}
\end{table}
Mithilfe dieser targets können beispielsweise voneinander abhängige Hintergrundservices nacheinander gestartet werden. Neben den runlevels entsprechenden targets gibt es noch einige weitere, wie zum Beispiel das \textbf{network-online.target}.
Auch der Raspberry Pi mit Raspbian verwendet systemd. Für die DeSearch-Boxen eignet es sich, um den Scan-Vorgang als Hintergrundprozess zu verwalten und so sicherzugehen, dass dieser stabil läuft und nach eventuellem reboot selbständig wieder startet.\newline
Über die Konsole kann \textbf{systemctl} aufgerufen werden. Dies ist die Überwachungs- und Steuereinheit von systemd. Mit \texttt{systemctl start xyz.service} kann ein Service gestartet werden. Respektive kann mit  \texttt{systemctl restart xyz.service} und \texttt{systemctl stop xyz.service} der Service neugestartet oder angehalten werden. Mit  \texttt{systemctl status xyz.service} wird eine Ausgabe erzeugt, die über den Status des Service informiert und eventuelle Ausgaben des Service anzeigt. Diese Kommandos und alle weiteren Optionen von systemctl befinden sich im Manual \citep[][]{systemctl-man}.

\subsection{Umsetzung der Anforderungen}

\subsubsection{User-Authentifizierung}
\subsubsection{Erkennen der Marken über Bluetooth}
Durch die eingschränkte Sendereichweite von Bluetooth soll die Ortung im DeSearch-Projekt gelingen. Die dementen Patienten tragen kleine BLE-Sendemarken am Körper, die von den fest installierten DeSearch-Boxen erkannt werden, sobald sich der Patient in der Nähe aufhält. So kann auf den ungefähren Aufenthaltsort der Person geschlossen werden, sobald eine der Boxen einen Treffer an die Zentrale meldet. \newline
Als Ergebnis eines studentischen Workshops auf der Hütte in Balderschwang gibt es eine Python Klasse, die einen BLE Scan starten kann und basierend auf einer Liste von MAC Adressen eine Callback-Funktion aufruft, wann immer eine der MAC Adressen in der Liste gesichtet wird.
Diese Klasse verwendet die Bibliothek "bleep", die sich allerdings im Laufe der Projektarbeit als nicht geeignet herausstellte, da sie folgende Probleme aufweist:
\begin{itemize}
	\item Wird sie nicht mit root-Rechten ausgeführt, kommt es zum Absturz.
	\item Die Installation ist für eine großflächige Verteilung zu aufwändig und erfordert unter anderem die Installation einer modifizierten Version von "pygattlib" \citep[Vgl.][]{bleep-installation}.
	\item Es handelt sich dabei um ein Einmannprojekt, dessen Entwickler die Bibliothek aus einem privaten Bedürfnis heraus geschaffen hat. Damit ist die Softwarequalität und Wartung abhängig von einer einzelnen Person.
\end{itemize}
Zur Bluetooth-Kommunikation unter Linux kann der im Linux Kernel enthaltene Bluetooth Stack "bluez" verwendet werden. Dieser enthält auch Treiber für die meisten Bluetooth Chipsets.
Auch die bleep Bibliothek verwendet bluez. Im Rahmen der Projektarbeit erfolgt eine Reimplementierung der Scanner-Klasse in Python, die ohne die Verwendung von bleep auskommt.
In der Reimplementierung wurden dabei die Schnittstellen der Klasse erhalten, die Kommunikation mit bluez findet nun über den DBus statt. Dies erfordert keine root-Rechte und ist neben den Kommandozeilen-Tools der einzig im offiziellen Git-Repository von bluez dokumentierte Weg zum Ansprechen der Schnittstellen \citep[Vgl.][]{bluez-git}. Außerdem werden damit nur noch die DBus-Bibliothek von Python und bluez auf den DeSearch-Boxen benötigt, diese sind in den offiziellen Paketquellen enthalten und damit einfach zu verteilen und zu aktualisieren.

\subsubsection{Robustheit des Systems durch Services}\label{sssec:services}
Um eine gewisse Robustheit des Systems zu gewährleisten, muss sichergestellt werden, dass die Python-Skripte zum Scan auf den DeSearch-Boxen sowie die Python-Skripte des Web-Servers auf der Zentrale immer laufen. Dies wird durch Hintergrundservices realisiert. Anforderung an die Services ist es, dass sie sich nach einem Reboot selbständig starten und nach einem unerwarteten Fehler oder Ausführungs-Stop selbständig re-spawnen(d.h. sich selbst neu-starten). Realisiert wurde dies durch systemd-Services, welche in Kapitel \ref{sssec:systemd} bereits erläutert wurden. Die Konfiguration innerhalb des Service ist für Zentrale und DeSearchBox gleich, da die Anforderungen auch gleich sind. Um einen Hintergrundservice anzulegen, muss zunächst das \texttt{.service}-File unter dem Pfad \texttt{/lib/systemd/system/} angelegt werden. In diesem Verzeichnis werden alle Services von systemd nachgeschlagen. Beispielhaft ist in Abbildung \ref{fig:service} das Service-File der DeSearch-Zentrale zu sehen.
\begin{figure}[bth]
	\centering
	\includegraphics[width=1.0\linewidth]{images/service-systemd}
	\caption[Service-File auf dem Zentral-Pi]{Service-File auf dem Raspberry Pi, der als Zentrale fungiert.}
	\label{fig:service}
\end{figure}
Zunächst werden unter \texttt{[Unit]} Informationen zu dieser Service-Unit aufgeführt. Dies ist die Beschreibung des Service (unter Description) und die Information, wann der Service beim Boot gestartet werden soll. Hier stehen unter Anderem die Optionen \enquote{Before} und \enquote{After} zur Verfügung. Im \texttt{desearch-server.service} wird durch \enquote{After=network-online.target} ausgedrückt, dass der Service erst gestartet werden darf, wenn eine aktive Netzwerkverbindung besteht (siehe auch Kapitel \ref{sssec:systemd}). Ansonsten wird der Start des Service verzögert, bis die angegebene Bedingung erfüllt ist.
In der darauf folgenden \texttt{[Service]}-Sektion wird angegeben, was der Hintergrundservice tun soll. Die Angabe des \enquote{simple}-Type bewirkt, dass der Prozess, den der Service startet, auch dessen Haupt-Prozess ist. Anders verhält es sich, wenn hier \enquote{forking} gewählt wird: Dann verhält sich der Prozess wie ein traditioneller UNIX-Daemon und startet als Kindprozess des Service. \\
Nach der Angabe des Typs wird unter \enquote{ExecStart} angegeben, was der Service ausführen soll. Hier wird zunächst der Pfad der Python-Installation angegeben und anschließend der Pfad des deSearch-Server-Skripts, das von Python ausgeführt werden soll. Zusätzlich wird noch ein Working-Directory angegeben, da Python sonst von dem Service im Root-Verzeichnis gestartet wird und dann die restlichen benötigten Python-Skripte nicht gefunden werden. \\
Der nächste Parameter, \enquote{Restart}, bestimmt wann der Service neu gestartet werden soll. Hier gibt es verschiedene Optionen, die in der man-Page von systemd nachgeschlagen werden können \citep{systemd-service}. Für lang laufende Hintergrundprozesse empfiehlt die man-Page die Wahl von \enquote{Restart=on-failure}. Der Service wird bei unsauberen Exit-Codes (ungleich 0), wenn er von einem Signal beendet wird (z.B. Core Dump), bei einem Timeout oder (falls konfiguriert) bei fehlendem Watchdog-Signal neu gestartet. Die maximale Laufzeit des Service wird zudem im nächsten Argument \enquote{RuntimeMaxSec} auf unendlich gesetzt.\\
Im \texttt{[Install]}-Abschnitt wird abschließend festgelegt, welche Abhängigkeiten ein Service hat. Die Angabe \enquote{multi-user.target} stellt eine Abhängigkeit her, die bewirkt, dass der Service dann gestartet wird, wenn auch die Multi-User-Umgebung gestartet wird. \\
Nach dem Anlegen des Service-Files wird \texttt{systemctl daemon-relaod} einmalig ausgeführt, um anschließend mit \texttt{systemctl enable desearch-server.service} den Service zu aktivieren. Nun kann entweder durch einen reboot der Service automatisch gestartet werden (wie konfiguriert) oder mittels \texttt{systemctl start desearch-server.service} auch manuell. Abbildung \ref{fig:systemctl-status} zeigt die Statusausgabe von Systemctl nach erfolgreichem Start des Service.
\begin{figure}[bth]
	\centering
	\includegraphics[width=1.0\linewidth]{images/service-status}
	\caption[Status-Ausgabe von Systemctl]{Status-Ausgabe von Systemctl für den deSearch-Zentral-Service.}
	\label{fig:systemctl-status}
\end{figure}
\subsubsection{Datenbankschema}
In Abbildung \ref{img:db-schema} ist das Datenbankschema zu sehen, mit dem die Zentrale arbeitet. 
\begin{figure}
	\centering
	\includegraphics[width=1.0\linewidth]{images/db-schema}
	\caption[Datenbankschema der Zentrale]{Datenbankschema der Zentrale, Quelle: eigene Darstellung}
	\label{img:db-schema}
\end{figure}
\\Für den Zugang zur Web-Oberfläche werden \textbf{Benutzer} mit ihren Rollen in der Datenbank angelegt. Das Passwort wird dabei als Hash mit 128 Zeichen in der Datenbank gespeichert. 
\\Jede \textbf{DeSearch-Box} wird in der Datenbank mit ihrer Position und Boxnummer angelegt. Für die Zukunft wurde bereits das Feld \enquote{Boxtyp} erstellt, da im weiteren Verlauf ortsfeste, mobile und Alarm-Boxen unterschieden werden sollen. Bei mobilen Boxen(z.B. im Auto, in Linienbussen) wird anstatt einer festen Positionsbeschreibung die genaue Geo-Location mit Longitude und Latitude ermittelt. Alarm-Boxen können beispielsweise am Hof-Ausgang oder an der Bushaltestelle installiert werden und standardmäßig immer Alarm auslösen, wenn dort eine Marke gefunden wird.
\\Die \textbf{Marken} werden in der Datenbank mit einer eindeutigen Mac-Adresse abgelegt, die beim Abgleich mit den Scan-Ergebnissen verwendet wird. Außerdem wird die ID des zugehörigen Patienten vermerkt und ein Hinweis auf das jeweilige Kleidungsstück, in das die Marke eingenäht ist. Zusätzlich wird ein Ablaufdatum in die Datenbank eingetragen, das den ungefähren Zeitpunkt des nächsten Batteriewechsels festhält. Somit kann eine Warnmeldung angezeigt werden, wenn der Batteriewechsel fällig ist. Ein wichtiges Feld ist der boolean \enquote{wird\_gesucht}. Wird auf der Web-Oberfläche eine Suche gestartet, so werden alle zu dieser Person gehörigen Marken auf wird\_gesucht gesetzt. Nur dann werden Treffer zu dieser Person von den Boxen an die Zentrale gemeldet. Finden die Suchenden beispielsweise nur ein Kleidungsstück der vermissten Person, können einzelne Marken gezielt von der Suche ausgeschlossen werden. 
\\Ein \textbf{Patient} liegt in der Datenbank mit Name, Vorname und weiteren für die Suche wichtigen Informationen vor. Beispielsweise kann unter Merkmale die Beschreibung einer Person vermerkt werden, unter Medikation und Anmerkungen können wichtige Hinweise im Fall eines Fundes oder einer längeren Abwesenheit der Person gespeichert werden. Gerade bei Diabetikern oder Patienten, die regelmäßig Medikamente nehmen müssen, ist diese Information besonders wichtig. Das Flag \enquote{Auf\_Ausgang} kann gesetzt werden, falls ein Angehöriger mit der Person das Pflegeheim verlässt, beispielsweise für einen Spaziergang. Somit wird ein Fehlalarm verhindert.
\\Die Tabelle \textbf{GefundeneMarken} kann eine Suchaktion protokollieren. Jeder Treffer, den eine Box registriert und an die Zentrale sendet, wird hier dokumentiert. Ein Treffer wird mit der zugehörigen BoxID, der MarkenID, dem Zeitpunkt und der Position in die Datenbank gelegt. Somit können auf der Web-Oberfläche alle Treffer in einer Zeitreihe angezeigt werden. Aus Datenschutzgründen wird diese Tabelle nach jeder abgeschlossenen Suchaktion gelöscht.

\subsubsection{Administrative Benutzeroberfläche}
\subsubsection{Installation der DeSearch-Boxen in der Testumgebung}
Die Raspberry Pi's werden in der DHBW Ravensburg am Campus Friedrichshafen zum Testbetrieb installiert. Dabei fungiert ein Pi als Zentrale und die anderen als DeSearch-Boxen, die die Funde an die Zentrale melden. In Tabelle \ref{tab:pis} ist eine Übersicht über alle installierten Raspberry Pi's aufgelistet.
Dazu wurde vom Netzwerkadministrator der DHBW die Erlaubnis eingeräumt, die vorhandene Infrastruktur zu verwenden. Daran geknüpft sind allerdings folgende Bedingungen bzw. Einschränkungen:
\begin{itemize}
	\item Zugriffe aus dem Internet sind nicht erlaubt.
	\item Alle Pi's außer der Zentrale (Slaves) verbinden sich über WLAN unter Verwendung der persönlichen Zugangsdaten mit dem Netzwerk.
	\item Die Zentrale bekommt einen LAN Anschluss und eine feste IP Adresse plus DNS Eintrag zugewiesen.
	\item Ein Verbindungsaufbau zu den Slaves ist nicht möglich - jegliche Kommunikation muss von ihnen initiiert werden.
\end{itemize}
Die letzte Bedingung stellt eine Herausforderung dar, da damit auch kein SSH-Zugriff auf die DeSearch-Boxen möglich ist, der benötigt wird um die DeSearch Anwendung zu aktualisieren, starten oder überwachen.
Dieses Problem kann gelöst werden, da mit der Zentrale ein Gerät im Netzwerk ist, das für alle anderen Pi's und auch die zur Entwicklung und Wartung verwendeten Laptops erreichbar ist. Somit kann über die Zentrale ein Tunnel aufgebaut werden.
Zum Aufbauen des Tunnels bietet sich die Verwendung der SSH Portforwarding Funktion an, da alle beteiligten Geräte bereits für die Verwendung von SSH eingerichtet sind. Eine schematische Darstellung der Vorgehensweise ist in Abbildung \ref{fig:tunnel} zu sehen.
Zum Verbindungsaufbau wurde ein technischer Benutzer mit dem Namen "tunnel" auf der Zentrale angelegt, der sich über SSH und Zertifikat Authentifizierung einloggen kann.
Jeder Slave erhält einen RSA Schlüssel, den er zum Login verwenden kann, und über einen Eintrag in der Crontabelle wird sichergestellt, dass der Tunnel immer offen gehalten wird.
Der Ausgangsport des Tunnels ist dabei für jede DeSearch-Box eindeutig. Soll nun eine Verbindung zu einer Box hergestellt werden, so muss erst eine SSH Verbindung zur Zentrale aufgebaut werden und dort eine SSH Verbindung zum Eingang des Tunnels (localhost plus Tunnel Ausgangport) hergestellt werden.
\begin{table}[h]
	\begin{tabular}{ | p{2,5cm} | p{2,5cm} | p{4cm} | p{6cm} |}
		\hline
		\textbf{Pi-Nummer} & \textbf{Funktion} & \textbf{Installationsort} &  \textbf{Erreichbarkeit} \\ \hline
		3 & Zentrale & Büro Herr Judt & statische IP 141.68.30.39 oder \mbox{Judt-Master.it.ba-ravensburg.de} \\ \hline
		5 & DeSearch-Box & Haupteingang oberhalb der Treppe & Tunnel über Zentrale, Port 19005 \\ \hline
		
	\end{tabular}
	\caption{Übersicht der Raspberry Pi's mit Funktion, Installationsort und Erreichbarkeit}
	\label{tab:pis}
\end{table}

\begin{figure}
	\centering
	\includegraphics[width=\textwidth]{./images/tunnel.png}
	\caption[Schematische Darstellung der Verwendung des SSH Tunnels]{\textbf{Schematische Darstellung der Verwendung des SSH Tunnels} - Beispielhaft ist hier eine Verbindung vom Administrator zur DeSearch-Box 6 dargestellt (roter Pfeil), Quelle: eigene Darstellung}
	\label{fig:tunnel}
\end{figure} 

