\section{Schlussteil}
\subsection{Zusammenfassung}
Die zunehmend alternde Gesellschaft stellt uns vor eine steigende Anzahl an Demenzpatienten. Das \enquote{Altern zu Hause} als Wunsch vieler Patienten bedeutet eine große Herausforderung für Angehörige und Pflegende. Ein Problem der Demenz ist die sogenannte Weglauftendenz (siehe Kapitel \ref{ssec:weglauftendenz}), die bei Angehörigen große Panik oder Angst vor dem Vermisstenfall auslösen kann. Maßnahmen entgegen einer Weglauftendenz stehen unter einem rechtlichen Problem, da vor allem Pflegeheime keine freiheitsentziehenden Maßnahmen an ihren Patienten anwenden dürfen. Die Schwierigkeit besteht also darin, die dementen Personen vor den Gefahren des Weglaufens zu schützen, ohne ihnen dabei die Freiheit zu entziehen. Diese Problemstellung war ein Anreiz dafür, das iCare-Projekt mit dem DeSearch-System ins Leben zu rufen. Das iCare-Projekt, das sich mit den Möglichkeiten von IoT-Techniken im Kontext des \enquote{Alterns zu Hause} beschäftigt, erforscht Technologien, die Alternden und Pflegenden den Alltag erleichtern. Unter dieser Zielsetzung ist auch das DeSearch-System als Teil von iCare entstanden, das die Positionsbestimmung von an Demenz erkrankten Menschen in Angriff nimmt.  Hierfür sollen kleine Bluetooth-Low-Energy-Sender in die Kleidung der betroffenen eingenäht werden. Im Vermisstenfall kann eine Suche im DeSearch-System ausgelöst und die Position der Marken über die DeSearch-Boxen bestimmt werden. Die Boxen führen einen Bluetooth-Scan durch und melden den Fund einer Marke mit ihrer Position in einer DeSearch-Zentrale. Somit können die Personen ungefähr geortet werden. Ziel dieser Projektarbeit war die Umsetzung eines Prototypen von DeSearch, welcher die Suche nach Dozenten und Studenten auf dem Campus der DHBW in Friedrichshafen ermöglicht.\\
Für die Umsetzung des Projekts wurden folgende methodische Schritte unternommen: Zunächst wurde eine Recherche zur Hardware-Tauglichkeit durchgeführt. Die Entscheidung fiel für den Raspberry Pi (siehe Kapitel \ref{sssec:raspberry}) als Rechner für die DeSearch-Boxen und die Zentrale, als eingesetzte Funktechnologie wurde Bluetooth Low Energy gewählt (siehe Kapitel \ref{sssec:BLE}). Eine kleine Stückzahl von 14 Exemplaren des Raspberry Pi wurde für das prototypische System bestellt. Die Ergebnisse aus der Vorlesung \enquote{Web-Engineering 2} auf der Hütte in Balderschwang, bei der eine erste Implementierung des Systems umgesetzt wurde, flossen in die Umsetzung mit ein. Anschließend wurde der Prototyp in der Testumgebung installiert und iterativ weiterentwickelt. Ausführliche Testläufe im prototypischen Rahmen ergaben neue Erkenntnisse für eine spätere Produktivnutzung des Systems.\\
Die zuvor an das Projekt gestellten Erwartungen (siehe Kapitel \ref{sec:evaluierung}) wurden anhand des Prototyps überprüft und teilweise erfüllt. Nicht umgesetzte Kriterien überschritten entweder den Zeitrahmen des Projekts oder wurden als unkritisch für den Prototypenbau angesehen.
\subsection{Ausblick}
Das Prototypische System, das momentan funktionsfähig in der DHBW installiert ist, bleibt auch für zukünftige Weiterentwicklungen erhalten. Die Übergabe an Prof. Dr. Andreas Judt erfolgte am 22.06.2016. Die Benutzeroberfläche, die bei dieser Studienarbeit nicht im Vordergrund stand, sollte vor dem Produktiv-Einsatz überarbeitet und gegen Falscheingaben abgesichert werden.
In einem nächsten Schritt werden die Ergebnisse der Projektarbeit zusammen mit dem Partnerprojekt zum Entwurf der Bluetooth-Marken (Lisa Rudolf und Sandra Minsch, TIT13) der kooperierenden Universität St. Gallen vorgestellt. Für erste Produktivtests wurde bereits ein Kontakt zum Seniorenzentrum Altenheimat in Bondorf hergestellt. Dort sollen zunächst nur wenige Personen mit den Marken ausgestattet werden, um in enger Zusammenarbeit mit dem Pflegepersonal weitere Verbesserungen des Systems zu erarbeiten.  
