\section*{Abstract}
Die zunehmend alternde Gesellschaft stellt uns vor eine steigende Anzahl an Demenzpatienten. Das \enquote{Altern zu Hause} als Wunsch vieler Patienten bedeutet eine große Herausforderung für Angehörige und Pflegende. Ein Problem der Demenz ist die sogenannte Weglauftendenz, die bei Angehörigen große Panik oder Angst vor dem Vermisstenfall auslösen kann. Diese Problemstellung war ein Anreiz dafür, das iCare-Projekt mit dem DeSearch-System ins Leben zu rufen.\\
Das iCare-Projekt, das sich mit den Möglichkeiten von IoT-Techniken im Kontext des \enquote{Alterns zu Hause} beschäftigt, erforscht Technologien, die Alternden und Pflegenden den Alltag erleichtern. Unter dieser Zielsetzung ist auch das DeSearch-System als Teil von iCare entstanden, das die Positionsbestimmung von an Demenz erkrankten Menschen in Angriff nimmt. Im Vermisstenfall kann eine Suche im DeSearch-System ausgelöst und die Position der vermissten Person über Bluetooth bestimmt werden. Die Zielsetzung dieser Projektarbeit war die Umsetzung eines Prototypen von DeSearch, welcher die Suche nach Dozenten und Studenten auf dem Campus der DHBW in Friedrichshafen ermöglicht. 
Für die Umsetzung des Projekts wurden Hardware-Entscheidungen getroffen, Prototypen implementiert und das System in der DHBW Friedrichshafen ausgiebig getestet. In Zukunft soll der Prototyp in Zusammenarbeit mit der Internationalen Bodenseehochschule weiter verbessert sowie in einem Pflegezentrum installiert und getestet werden.
\begin{center}
	\rule{0.3\textwidth}{0.4mm}
\end{center}
Rising numbers of patients with dementia demand new strategies and solutions in elder care. Many patients wish to \enquote{age in place}, at home. One problem of people with dementia is their tendency to run away and get lost. To solve this problem, the \enquote{iCare}-Project was initiated. This Project drives research on how IoT-Technologies help in the daily routine of caregivers or relatives. The \enquote{DeSearch}-System as a part of iCare targets the location determination of persons with dementia. In a missing person case, a search can be triggered in the DeSearch-System, which locates the missing person via Bluetooth. The objective of this student research project was the implementation and realization of a prototype which can be tested in the area of DHBW Friedrichshafen. The execution included hardware decisions, software implementation and testing in the campus area. In the foreseeable future, the prototype will be transferred to a retirement home, where productive testing will be continued. 