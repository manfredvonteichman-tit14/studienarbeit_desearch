
\section{Einleitung}
\subsection{Hinführung}
In nahezu jedem Land der Erde wächst der Anteil der Menschen über 60 schneller an als jede andere Altersgruppe der Bevölkerung. Mit der alternden Bevölkerung steigt auch die Zahl der Demenz-Erkrankungen. In Deutschland lebten 2014 1,5 Millionen Demenzkranke, jährlich treten ca. 300.000 Neuerkrankungen auf. Aufgrund des demografischen Gesellschaftswandels kommt es zu mehr Neuerkrankungen als Sterbefällen, wodurch die Zahl der Demenzpatienten kontinuierlich ansteigt \citep[Vgl.][]{demenz}.\\
Zeitgleich zeichnet sich eine weitere Entwicklung in unserer Gesellschaft ab: Das \enquote{Internet of Things} wird allgegenwärtig. Wir umgeben uns mit elektronischen Geräten, Sensoren, Überwachungssystemen, und lassen uns gerne und oft von smarten Gegenständen und dem Internet helfen. Möglichkeiten, die vor kurzer Zeit noch nur als Gedankenexperiment existierten, lassen sich nun umsetzen. Das Zusammenspiel von Sensoren und Datenanalysen und die Kommunikation vieler Geräte untereinander rückt auch bei medizinischen Anwendungen immer mehr in den Fokus. Überwachung, Diagnose und medizinische Versorgung können durch das IoT unterstützt und vereinfacht werden. In der folgenden Arbeit sollen IoT-Technologien und die Problemstellungen bei der Pflege von Demenzpatienten zusammengeführt werden.
\subsection{Problemstellung}
Bei an Demenz erkrankten Menschen besteht das Risiko einer Weglauftendenz. Das bedeutet, dass diese ohne das Wissen von Angehörigen oder Pflegern beschließen, ihr Haus oder das Pflegeheim zu verlassen. Vom Zeitpunkt des Verschwindens bis jemand dieses bemerkt können kritische Stunden vergehen, in denen die Betroffenen in einen Bus einsteigen, stürzen oder unterkühlen können. Auch das Verschwinden im Pflegeheim selbst kann schon zur Gefahr werden, wenn sich die Betroffenen beispielsweise stundenlang im Keller aufhalten. Ist ein Demenzpatient verschwunden, bricht bei Angehörigen oder Pflegern Panik aus. Suchaktionen, nicht selten mit der Polizei zusammen, werden gestartet. Bei der Überlegung, wie man solchen Notfallsituationen vorbeugen kann, stehen die Pflegenden vor einem Dilemma. Jeder Patient hat das Recht auf uneingeschränkte Bewegunsfreiheit, durch Freiheitsentzug könnte das Problem des Weglaufens jedoch gelöst werden. Auch Sicherungssysteme wie Ortungschips sind rechtlich umstritten. Laut Pflege-Qualitätssicherungsgesetz gelten sie als freiheitsentziehende Maßnahme und bedürfen der richterlichen Genehmigung \citep[Vgl.][]{pqsg}. Andererseits existieren auch Urteile, die das Anbringen von Ortungssystemen ohne gerichtliche Zustimmung erlauben\footnote{OLG Brandenburg, 19.01.2006 - 11 Wx 59/05\\AG Coesfeld, 31.08.2007 - Az. 9 XVII 214/06}. Die Schwierigkeit besteht also darin, die dementen Personen vor den Gefahren des Weglaufens zu schützen, ohne ihnen dabei die Freiheit zu entziehen. Diese Problemstellung war ein Anreiz dafür, das iCare-Projekt mit dem DeSearch-System ins Leben zu rufen.
\subsection{Zielsetzung}
 Ziel von DeSearch ist es, den Standort der betroffenen Menschen im Vermisstenfall diskret bestimmen zu können. Dabei sollen der Aufenthaltsort der Personen keinesfalls dauerhaft getrackt werden, da dies rechtlich nicht tragbar wäre. Stattdessen soll bei einem gemeldeten Vermissten eine Suche ausgelöst werden, worauf das DeSearch-System den Standort der Person über Bluetooth bestimmt. Dies soll durch den Einsatz von kleinen Bluetooth-Marken, die unbemerkt in die Kleidung eingenäht werden können, realisiert werden. Die Marken werden von sogenannten DeSearch-Boxen erkannt, wenn sie sich in Sendereichweite der Boxen befinden, und können so einem Standort zugeordnet werden. Eine Zentrale, die vom Pflegepersonal bedient werden kann, dient dem Starten von Personensuchen und dem Anzeigen der Ergebnisse über eine Web-Oberfläche. Wird die Person nicht als vermisst gemeldet, können keinerlei Rückschlüsse auf deren Aufenthaltsort gezogen werden.
\\Zielsetzung dieser Projektarbeit ist es, einen lauffähigen Prototypen der DeSearch-Box zu bauen und mit den Bluetooth-Marken zu verknüpfen. Das erste Testsystem soll im Umfeld der DHBW Friedrichshafen aufgesetzt werden, und als “Dozentenfinder” getestet werden. Ziel dabei ist es, einen Dozenten mit Bluetooth-Marken und das DHBW-Gebäude mit den DeSearch-Boxen auszustatten. Es soll möglich sein, den Dozenten anhand der aufgestellten DeSearch-Boxen lokalisieren zu können und eine Aufzeichnung seiner Aufenthaltsorte zu generieren.
Im weiteren Verlauf soll der Projektumfang so ausgeweitet werden, dass das System in einem Seniorenzentrum der Evangelischen Altenheimat Bondorf im größeren Umfang getestet werden kann.

\subsection{Methodisches Vorgehen zur Problemlösung}
Die einzelnen Schritte der Studienarbeit werden sein:
\begin{itemize}
\item Recherche zur Tauglichkeit verschiedener Hardware-Varianten
\item Hardware-Entscheidung
\item Hardware-Beschaffung (kleine Stückzahl zum Anfertigen eines Prototypen)
\item Bau eines Prototypen
\item Aufbau der Systemarchitektur und Vernetzung(im Hüttenworkshop in Zusammenarbeit mit dem ganzen Kurs)
\item Aufbau des Systems in der DHBW Friedrichshafen
\item Testläufe mit Dozentenfinder
\item Aufbau und Einrichten des Testsystems im Seniorenzentrum Altenheimat Bondorf
\end{itemize}



\subsection{Erwartungen an die Lösungsergebnisse}
Folgende Anforderungen sollten durch den ersten Prototypen erfüllt sein:
\begin{itemize}
	\item  Absicherung der Daten und der DeSearch-Boxen, beispielsweise durch Zertifikate, zum Schutz gegen Manipulation
	\item Authentifizierung der Administratoren auf den DeSearch-Boxen und der Zentrale 
	\item Periodisches Anfragen der DeSearch-Boxen bei der Zentrale, welche Marken momentan gesucht werden
	\item Meldung der DeSearch-Box bei der Zentrale, wenn eine Marken gesehen wurde mit: Marken-ID, Box-ID, Signalstärke, Zeitstempel
	\item Starten einer Suche über die Benutzeroberfläche der Zentrale
	\item Anzeigen der Suchergebnisse auf der Benutzeroberfläche
	\item Eine Person liegt in der Datenbank vor mit: Vorname, Nachname, ID, Anmerkungen zur Person (beispielweise Medikamente, Verhalten, Berduerfnisse,...), gesucht-Flag
	\item Zuordnung einer Personen-ID zu den MAC-Adressen der Marken, sodass einer Person mehrere Marken zugeordnet werden können
	\item Hinzufügen, Bearbeiten und Entfernen von Mitarbeitern, Personen und Marken in der Datenbank, Bedienung über die Benutzeroberfläche der Zentrale
\end{itemize}

\nomenclature{ID}{\textbf{Id}entifikator}
\nomenclature{MAC-Adresse}{\textbf{M}edia-\textbf{A}ccess-\textbf{C}ontrol-Adresse}

\subsection{Überprüfung der Ergebnisse}
Die Überprüfung der Ergebnisse soll zum einen anhand der gestellten Anforderungen erfolgen. Die ausführlichen Geschäfts- und Systemfälle werden einzeln durchgespielt und auf Erfüllung überprüft. Eine Auflistung der vor Projektbeginn spezifizierten Systemfälle ist in Anhang \ref{anh:fälle} zu sehen. Zusätzlich zu diesen Spezifikationen wird die Installation in der Testumgebung und der ausführliche Testbetrieb zur Überprüfung der Ergebnisse genutzt. Details zur Evaluierung der Ergebnisse finden sich in Kapitel \ref{sec:evaluierung}.









