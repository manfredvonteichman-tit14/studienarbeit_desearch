\section{Evaluierung der Lösungsergebnisse}\label{sec:evaluierung}
Die Lösungsergebnisse sollen im Folgenden Kapitel anhand fester Kriterien überprüft werden. Ein Kriterienkatalog wurde aus den Geschäfts- und Systemfällen entwickelt, die als Ergebnis aus einer studentischen Veranstaltung zu Projektbeginn entstanden sind (siehe Anhang \ref{anh:fälle}).
Anhand der Kriterien in Tabelle \ref{tab:kriterien} soll die Problemlösung überprüft werden. Alle Kriterien werden in der Testumgebung an der DHBW Friedrichshafen durchgespielt und anschließend bewertet.
\begin{center}
	\begin{longtable}{ | p{2,5cm} | p{5cm} | p{5cm} | p{2,5cm} |}
	 \hline
	 \textbf{Kriterium} & \textbf{Beschreibung} & \textbf{Erwartetes Ergebnis} & \textbf{Erfüllungs-grad} \\ \hline
	 Vermisster Patient wird erfasst & Ein Patient wurde in der Zentrale als vermisst markiert und die entsprechende Marke befindet sich in Reichweite einer DeSearch-Box & Die Box erkennt die entsprechende Marke und meldet sich bei der Zentrale & erfüllt; Im Test je nach Bluetooth-Gerät (Marke, Smartphone) unterschiedlich zuverlässig erkannt \\ \hline
	 Andere Marken werden ignoriert & Ein Patient wurde in der Zentrale als vermisst markiert und Marken von anderen Patienten befinden sich in Reichweite einer DeSearch-Box & Die Box erkennt, dass die Marken in Reichweite nicht vermisst werden und meldet nichts zurück & erfüllt; Im Test kein einziger false positive \\ 
	 \hline
	 DeSearch-Box wird konfiguriert & Eine neue DeSearch-Box soll konfiguriert bzw. eine bestehende Box soll angepasst werden & Konfigurierte und im System verbundene DeSearch-Box & erfüllt; Siehe Kapitel \ref{sssec:schritte} \\ 
	 \hline
	 DeSearch-Box wird befestigt & Eine DeSearch-Box soll am Einsatzort befestigt werden & Vor Vandalismus und Diebstahl gesicherte, im Netzwerk erreichbare Box mit permanenter Stromversorgung & nicht zutreffend für Prototyp, Anforderung an Produktivsystem \\ 
	 \hline
	 temporärer Ausfall DeSearch-Box & Ein unerwarteter Ausfall der Box tritt zeitweise auf & Nach Reboot der Box soll sich diese selbständig wieder im System anmelden und den Scan-Vorgang wieder aufnehmen & erfüllt durch Services und cronjob, siehe Kapitel \ref{sssec:services}\\ 
	 \hline
	 permanenter Ausfall DeSearch-Box & Ein unerwarteter Ausfall der Box tritt permanent auf, die Box sendet keine alive-Meldungen mehr & Das System muss den Ausfall feststellen, sodass die Box manuell wieder im System reaktiviert werden kann & erfüllt; In der UI kann Ausfall festgestellt werden(siehe Kapitel \ref{sssec:ui}) \\ 
	 \hline
	 DeSearch-Box fordert gesucht-Liste an & Die Box fordert periodisch die aktuelle Liste der gesuchten Marken bei der Zentrale an & Die Zentrale übermittelt der Box die aktuelle Liste der gesuchten Marken & erfüllt, Anfrage alle 10 Sekunden \\ 
	 \hline
	 App fordert gesucht-Liste an & Die Benutzeroberfläche fordert beim Start der Applikation die aktuelle Liste der gesuchten Personen an & Die Zentrale übermittelt der App die aktuelle Liste der gesuchten Personen & erfüllt, siehe Kapitel \ref{sssec:ui}\\ 
	 \hline
	 Mitarbeiter deaktiviert Marke & Eine gesuchte Marke wurde gefunden, aber die gesuchte Person trägt dieses Kleidungsstück momentan nicht & In der App wird die Marke deaktiviert, die Zentrale entfernt die Marke von der gesucht-Liste & erfüllt, siehe Kapitel \ref{sssec:ui} \\ 
	 \hline
	 Person wurde gefunden & Ein Mitarbeiter beendet die Personensuche in der App & Die Zentrale entfernt alle entsprechenden Marken von der gesucht-Liste & erfüllt, siehe Kapitel \ref{sssec:ui}\\ 
	 \hline
	 
	 \caption{Erfüllungskriterien zur Überprüfung der Problemlösung}
	
	\label{tab:kriterien}
	\end{longtable}
	
\end{center}

